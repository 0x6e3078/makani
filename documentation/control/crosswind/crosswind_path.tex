\documentclass{article} %[11pt]{amsart}
\usepackage{geometry}
\geometry{letterpaper}
\usepackage{graphicx}
\usepackage{amssymb}
\usepackage{amsmath}
\usepackage{siunitx}
\usepackage{tikz}
\usepackage{lscape}
\usepackage{pgfplots}
\usetikzlibrary{
  arrows,
  calc,
  decorations.markings,
  decorations.pathreplacing,
  dsp,
  fit,
  positioning
}
\makeatletter
\dspdeclareoperator{dspvoidshapeadder}{
  % Coordinate offset for the plus
  \pgfutil@tempdima=\radius
  \pgfutil@tempdima=0.55\pgfutil@tempdima
  \pgfusepathqstroke
}
\tikzset{
  vdspadder/.style={
    shape=dspvoidshapeadder,
    line cap=rect,
    line join=rect,
    line width=\dspblocklinewidth,
    minimum size=\dspoperatordiameter,
    label={185:$+$},
    label={265:$-$}
  },
  vadspadder/.style={
    shape=dspvoidshapeadder,
    line cap=rect,
    line join=rect,
    line width=\dspblocklinewidth,
    minimum size=\dspoperatordiameter,
    label=below right:$-$,
    label=above right:$+$
  },
  vkdspadder/.style={
    shape=dspvoidshapeadder,
    line cap=rect,
    line join=rect,
    line width=\dspblocklinewidth,
    minimum size=\dspoperatordiameter,
    label=below right:$-$,
    label=below left:$+$
  },
  vjdspadder/.style={
    shape=dspvoidshapeadder,
    line cap=rect,
    line join=rect,
    line width=\dspblocklinewidth,
    minimum size=\dspoperatordiameter,
    label=above right:$+$,
    label=below right:$-$
  },
  edspadder/.style={
    shape=dspvoidshapeadder,
    line cap=rect,
    line join=rect,
    line width=\dspblocklinewidth,
    minimum size=\dspoperatordiameter
  }
}
\makeatother


\tikzset{dspblock/.style={inner xsep = 5pt, inner ysep = 5pt}}
\tikzset{dspblock2/.style={inner xsep = 5pt, inner ysep = 10pt}}
\tikzset{dspblock3/.style={inner xsep = 5pt, inner ysep = 20pt}}


\pgfplotsset{grid style={dashed,gray!25}}

\newcommand{\qhat}{\hat{q}}
\newcommand{\cbar}{\bar{c}}
\newcommand{\qbar}{\bar{q}}
\newcommand{\cmd}{\mathrm{cmd}}
\newcommand{\ff}{\mathrm{ff}}
\newcommand{\eff}{\mathrm{eff}}
\newcommand{\app}{\mathrm{app}}
\newcommand{\wind}{\mathrm{wind}}
\newcommand{\kite}{\mathrm{kite}}
\newcommand{\nom}{\mathrm{nom}}
\newcommand{\aero}{\mathrm{aero}}
\newcommand{\geom}{\mathrm{geom}}

\title{Linear analysis of the path controller}
\date{DRAFT\\Dec 25, 2015}

\begin{document}
\maketitle

\section{Description of the path controller}

The path controller flies the kite along a set, usually circular,
path.  Its main inputs are the current position and velocity of the
kite and its main output is a curvature command, which the inner loop
of the crosswind controller attempts to meet.

The path controller uses a simple kinematic model of the kite's flight
path, which assumes that the kite, starting from its current position
and velocity, follows its current curvature for a period $t_1$ and
then follows a new commanded curvature for a period $t_2$ (see
Fig. \ref{fig:vector_field}).  The separation of the path into current
and commanded curvatures is motivated by the fact that the kite cannot
instantly change its curvature.  Multiple potential commanded
curvatures are projected forward and the one that minimizes a specific
cost function is chosen.

The cost function is the negative of the dot product of the predicted
velocity with a vector defined by the predicted position and a vector
field of ideal velocities.  These ideal velocities are tangent to the
path on the path, while away from the path they steer the kite back
toward the path assuming some minimum turning radius,
$r_{\mathrm{min}}$.

The exact equations defining the vector field are
\begin{eqnarray}
x &=& \frac{\pi}{2} \frac{\delta}{r_{\mathrm{min}}} \\
\vec{v} &=&\left\{
\begin{array}{lrcl}
\hat{e}_r  &                      &\delta& <  -r_{\mathrm{min}} \\
-\sin x \cdot \hat{e}_r - \cos x \cdot \hat{e}_{\theta}
  &\;\;\;\; -r_{\mathrm{min}} \le &\delta& \le r_{\mathrm{min}} \\
-\hat{e}_r & r_{\mathrm{min}} <   &\delta&
\end{array}
\right.
\end{eqnarray}
where $\hat{e}_r$ and $\hat{e}_{\theta}$ are the radial and tangential
unit vectors.

\begin{figure}
\begin{center}
\begin{tikzpicture}[scale=2.5]
  \def\R{1};
  \def\rmin{0.7};
  \def\r{0.1};

  % Draw vector field.
  \begin{scope}[color=lightgray]
    \foreach \x in {-2, -1.79, ..., 2} {
      \foreach \y in {-2, -1.79, ..., 2} {
        \def\norm{max(sqrt(\x*\x + \y*\y), 0.1)};
        \def\radx{(\x / \norm)};
        \def\rady{(\y / \norm)};
        \def\tanx{(\y / \norm)};
        \def\tany{(-\x / \norm)};
        \def\ang{max(min(0.5 * pi * (\norm - \R) / \rmin, pi/2), -pi/2)};
        \def\crad{(-0.1 * sin(deg(\ang)))};
        \def\ctan{( 0.1 * cos(deg(\ang)))};
        \draw [arrows={-latex}] (\x, \y) --
              ({\x + \crad * \radx + \ctan * \tanx},
               {\y + \crad * \rady + \ctan * \tany});
      };
    };
  \end{scope}
  \draw [arrows={latex-latex}] (-2, 0) -- (2, 0);
  \draw [arrows={latex-latex}] (0, -2) -- (0, 2);
  \draw [color=black] (0, 0) circle (\R);
  \draw [color=black, dashed] (0, 0) circle ({\R - \rmin});
  \draw [color=black, dashed] (0, 0) circle ({\R + \rmin});

  \draw [decorate,decoration={brace,amplitude=5pt}]
        ({(\R + \rmin) * cos(30)}, {(\R + \rmin) * sin(30)}) --
        ({\R * cos(30)}, {\R * sin(30)})
        node[midway, below right] {$r_{\mathrm{min}}$};

  % Draw trajectory.
  \def\startangle{60}
  \def\etaone{30}
  \def\epsangle{15};
  \def\headingangle{(\startangle - \epsangle)}
  \coordinate (start) at ({-(\R + \r) * cos(\startangle)},
                          { (\R + \r) * sin(\startangle)});
  \draw [thick, color=black]
        (start)
        arc ({180 - \startangle + \epsangle}:
             {180 - \startangle + \epsangle - \etaone/2}:
             {\R})
        node [above=0.5cm, left] {$t_1$}
        arc ({180 - \startangle + \epsangle - \etaone/2}:
             {180 - \startangle + \epsangle - \etaone}:
             {\R})
        coordinate (midpoint);

  \draw [thick, color=black, arrows={-latex}]
        (start)
        -- ($(start) + ({0.5 * sin(\headingangle)}, {0.5 * cos(\headingangle)})$);

  \foreach \command in {0.6, 0.8, ..., 1.4} {
    \draw [color=blue, arrows={-latex}]
          (midpoint)
          arc ({180 - \startangle + \epsangle - \etaone}:
               {180 - \startangle + \epsangle - \etaone - 50/\command}:
               {\command});
  }
  \path [color=blue]
        (midpoint)
        arc ({180 - \startangle + \epsangle - \etaone}:
             {180 - \startangle + \epsangle - \etaone - 25 / 1.4}:
             {1.4})
        node [above] {$t_2$};
\end{tikzpicture}
\caption{Vector field of ideal headings and the predicted trajectory
  of the kite.  The ideal headings are based around returning to the
  circle path using a minimum turning radius, $r_{\mathrm{min}}$.}
\label{fig:vector_field}
\end{center}
\end{figure}

\section{Kinematics of circular path following}

\begin{figure}[!ht]
\begin{center}
  \begin{tikzpicture}
    \def\R{4}
    \def\axislen{\R * 1.5}
    \def\deltaoffset{\R * 0.2}
    \def\deltar{\R * 0.1}
    \def\epsilondeg{20}
    \def\epsilonrad{\epsilondeg * 3.14159 / 180.0}
    \def\etadeg{50}
    \def\etarad{\etadeg * 3.14 / 180}

    \coordinate (pathorigin) at (0, 0);
    \coordinate (turnorigin) at
                ({-\deltaoffset - \R + (\R + \deltar) * cos(\epsilondeg)},
                 {(\R + \deltar) * sin(\epsilondeg)});
    \coordinate (nominalturnorigin) at
                ({-\deltaoffset - \R + \R * cos(\epsilondeg)},
                 {\R * sin(\epsilondeg)});
    \coordinate (nominalstart) at (-\R, 0);
    \coordinate (start) at (-\deltaoffset - \R, 0);

    % Draw path circle and axes.
    \draw [arrows={latex-latex}] (-\axislen, 0) -- (\axislen, 0);
    \draw [arrows={latex-latex}] (0, -\axislen) -- (0, \axislen);
    \draw (0, 0) circle (\R);

    % Draw turn circle.
    \draw[fill] (turnorigin) circle (0.05);
    \node[below, xshift=5pt] at (turnorigin) {$(x_c, y_c)$};
    \draw [color=gray!25] (nominalturnorigin) circle (\R);
    \draw [color=gray!60] (turnorigin) circle (\R + \deltar);

    \draw [arrows={-latex}] (turnorigin)
          -- ($(turnorigin) + ({(\R + \deltar) * cos(45)},
               {(\R + \deltar) * sin(45)})$)
          node[midway, below right] {$R + \delta r$};

    % Draw turn tradjectory.
    \draw[fill] (start) circle (0.05);
    \node[below left] at (start) {$(x_0, y_0)$};
    \draw (start)
          arc ({180 + \epsilondeg}:{180 + \epsilondeg - \etadeg}:{\R + \deltar})
          coordinate (end);

    \draw [dashed] (end) -- (pathorigin);
    \draw [dashed] (end) -- (turnorigin);
    \draw [dashed] (start) -- (turnorigin);
    \draw[fill] (end) circle (0.05);
    \node[above left] at (end) {$(x(\eta), y(\eta))$};

    \draw [decorate,decoration={brace,amplitude=5pt}]
          (nominalstart) -- (start)
          node[midway, below, yshift=-5pt] {$\delta_0$};

    \coordinate (nominalend) at ($(pathorigin)!\R cm!(end)$);
    \draw [decorate,decoration={brace,amplitude=5pt}]
          ($(nominalend)$) -- ($(end)$)
          node[midway, below left, yshift=-5pt] {$\delta(\eta)$};

    \draw [arrows={-latex}] (pathorigin) -- ({\R * cos(30)}, {\R * sin(30)})
          node[midway, below right] {$R$};

    % Label epsilon angles.
    \def\epsilonmarkerfrac{0.3}
    \draw [dashed] (start) --
          ($(start) + ({-\R/2 * sin(\epsilondeg)}, {\R/2 * cos(\epsilondeg)})$);
    \draw [dashed] (start) -- ($(start) + (0, \R/2)$);
    \draw [arrows={latex-latex}] ($(start) + (0, \R/4)$)
          arc (90:{90 + \epsilondeg}:{\R/4})
          node[left] {$\epsilon_0$};
    \path (start) -- ($(start)!\epsilonmarkerfrac!(pathorigin)$)
          arc (0:{\epsilondeg/2}:{(\R + \deltar) * \epsilonmarkerfrac})
          node {$\epsilon_0$};

    % Label \epsilon(\eta) angle.
    \path (start)
          arc ({180 + \epsilondeg}:
               {180 + \epsilondeg - \etadeg + 5}:
               {\R + \deltar})
          coordinate (turntangent);
    \path [dashed] let \p{end}=(end) in
          (end) -- ($(end) + (\y{end}, -\x{end})$)
          coordinate (pathtangent);
    \draw [dashed] (end)
          -- ($(end)!1.5cm!(pathtangent)$)
          coordinate (epsetapathmarker)
          -- ($(end)!2.5cm!(pathtangent)$);
    \draw [dashed] (end)
          -- ($(end)!-1.5cm!(turntangent)$)
          coordinate (epsetaturnmarker)
          -- ($(end)!-2.5cm!(turntangent)$);
    \draw [arrows={latex-latex}] (epsetaturnmarker)
          arc ({90 + \epsilondeg - \etadeg}:
               {90 + \epsilondeg - \etadeg - 0.6 * \epsilondeg)}:
               1.5cm);
    \path [arrows={-latex}] (epsetaturnmarker)
          arc ({90 + \epsilondeg - \etadeg}:
               {90 + \epsilondeg - \etadeg - 0.3 * \epsilondeg)}:
               1.5cm)
          node [above left] {$\epsilon(\eta)$};

    % Label \eta angle.
    \def\etamarkerfrac{0.15}
    \path (turnorigin) -- ($(turnorigin)!\etamarkerfrac!(start)$)
          coordinate (etamarker);
    \path (etamarker)
          arc ({180 + \epsilondeg}:
               {180 + \epsilondeg - \etadeg / 2}:
               {(\R + \deltar) * \etamarkerfrac})
          node {$\eta$};
  \end{tikzpicture}
  \caption{Geometry of the curvature controller. The dark circle is
    the commanded path.  The lighter circle shows a constant radius
    path with radius $R + \delta r$, that starts at a small crosstrack
    offset, $\delta_0$, and heading offset, $\epsilon_0$.  This
    trajectory is projected forward to the point $(x(\eta), y(\eta))$.
    The lightest circle is a constant radius path with
    radius $R$ for comparison.}
  \label{fig:path_geometry}
\end{center}
\end{figure}
%
Figure \ref{fig:path_geometry} shows the geometry of the circular
flight path and the trajectory of a wing flying at constant curvature,
$(R + \delta r)^{-1}$, with a small crosstrack offset, $\delta_0$, and
heading offset, $\epsilon_0$.  The center of the kite trajectory is
given by:
\begin{alignat}{2}
x_c &= -R - \delta_0 + (R + \delta r) \cos \epsilon_0
& \; &\approx -\delta_0 + \delta r \\
y_c &= (R + \delta r) \sin \epsilon_0 & \; &\approx R \epsilon_0
\end{alignat}
%
To determine the kinematics, the crosstrack offset and heading offset
are expressed in terms of the $\eta$ angle parameter, which is related
to time as follows:
%
\begin{equation}
\eta = \frac{v t}{R + \delta r}
\end{equation}
%
A point on the trajectory projected out for a time $t$ is then:
\begin{eqnarray}
x(\eta) &=& x_c - (R + \delta r)
(\cos \eta \cos \epsilon_0 + \sin \eta \sin \epsilon_0) \\
y(\eta) &=& y_c + (R + \delta r)
(\sin \eta \cos \epsilon_0 - \cos \eta \sin \epsilon_0)
\end{eqnarray}
or approximately
\begin{eqnarray}
x(\eta) &\approx&
-R \cos \eta - \delta_0 + (1 - \cos \eta - \eta \sin \eta) \delta r -
R \sin \eta \cdot \epsilon_0 \\
y(\eta) &\approx& R \sin \eta + (\sin \eta - \eta \cos \eta) \delta r +
R (1 - \cos \eta) \epsilon_0
\end{eqnarray}
%
The crosstrack offset and heading offset at time $t$ are then:
%
\begin{eqnarray}
\delta(\eta) &=& \cos \eta \cdot \delta_0 + (1 - \cos \eta) \delta r +
R \sin \eta \cdot \epsilon_0
\label{eqn:delta_eta} \\
\epsilon(\eta) &=& -\frac{\sin \eta}{R} \delta_0 + \frac{\sin \eta}{R} \delta r +
\cos \eta \cdot \epsilon_0
\label{eqn:epsilon_eta}
\end{eqnarray}

The kinematics of the system may be found by first taking the small
$\eta$ approximation of Eq. \ref{eqn:delta_eta} and
Eq. \ref{eqn:epsilon_eta}.
%
\begin{eqnarray}
\delta(\eta) &\approx& \delta_0 + R \eta \epsilon_0 \\
\epsilon(\eta) &\approx& -\frac{\eta}{R} \delta_0 + \frac{\eta}{R} \delta r +
\epsilon_0
\end{eqnarray}
%
Then, taking the time derivative:
%
\begin{eqnarray}
\dot{\delta} &=& v \epsilon \\
\dot{\epsilon} &=& -\frac{v}{R^2} \delta - v \delta \kappa_{\mathrm{cmd}}
\end{eqnarray}
%
Here $\delta r$ is replaced by $\delta \kappa$ using the relation
$\delta \kappa = -\delta r / R^2$.  Finally, these equations are
combined to arrive at\footnote{There should be a $\frac{a}{v} \dot{\delta}$
on the right-hand side if it is not assumed that velocity is constant.}:
%
\begin{equation}
\ddot{\delta} = -\frac{v^2}{R^2} \delta - v^2 \delta \kappa_{\mathrm{cmd}}
\label{eqn:delta_diffeq}
\end{equation}

If the curvature command takes the form of the typical crosstrack PD
controller:
%
\begin{equation}
v^2 \delta \kappa_{\mathrm{cmd}} = k_p \delta + k_d v \epsilon
\end{equation}
%
then the closed-loop system is:
%
\begin{equation}
\ddot{\delta} + k_d \dot{\delta} + \left(\frac{v^2}{R^2} + k_p \right) \delta = 0
\end{equation}
%
The closed-loop system is stable if $k_d > 0$ and $k_p > -v^2/R^2$.
The characteristic frequency and damping ratio of this system are:
%
\begin{eqnarray}
\omega_0 &=& \sqrt{\frac{v^2}{R^2} + k_p} \label{eqn:omega0} \\
\zeta &=& \frac{k_d}{2 \sqrt{\frac{v^2}{R^2} + k_p}} \label{eqn:zeta}
\end{eqnarray}

\section{Analysis of basic path control ($t_1 = 0$)}

In the simplest form of the path controller, it is assumed that the
kite instantly meets the commanded curvature.  Thus, when predicting
the future position and heading of the kite, the time spent at the
current curvature is set to zero ($t_1 = 0$).

The curvature command is chosen to minimize a cost function, which we
define as the negative of the dot product between the final velocity
vector and the ideal velocity vector as defined by the vector field
(see Fig. \ref{fig:vector_field}).  More simply put, the command is
chosen to align the final velocity vector with the ideal velocity
vector as much as possible.  Obviously, this minimum occurs when these
vectors are aligned, or more specifically when:
%
\begin{equation}
\epsilon(\eta) = -k \delta(\eta)\;\;\;\; k = \frac{\pi}{2 r_{\mathrm{min}}}
\label{eqn:cost_func_solution}
\end{equation}
%
Thus Eq. \ref{eqn:delta_eta} and Eq. \ref{eqn:epsilon_eta} can be
combined with Eq. \ref{eqn:cost_func_solution} to find the analytic,
linearized form of the curvature command:
\begin{equation}
\delta \kappa_{\mathrm{cmd}} = \frac{1}{R^2} \left[
\frac{\cos \eta - \frac{\sin \eta}{kR}}
{1 - (\cos \eta - \frac{\sin \eta}{kR})} \delta +
\frac{\sin \eta + \frac{\cos \eta}{kR}}
{1 - (\cos \eta - \frac{\sin \eta}{kR})} R \epsilon
\right]
\end{equation}
The equivalent linear crosstrack PD gains are:
\begin{eqnarray}
k_p &=& \frac{v^2}{R^2}
\frac{\cos \eta - \frac{\sin \eta}{kR}}
{1 - (\cos \eta - \frac{\sin \eta}{kR})} \\
k_d &=& \frac{v}{R}
\frac{\sin \eta + \frac{\cos \eta}{kR}}
{1 - (\cos \eta - \frac{\sin \eta}{kR})}
\end{eqnarray}

The characteristic frequency and damping ratio can be expressed in
terms of $\eta$, $r_{\mathrm{min}}$, $v$, and $R$ by substituting
these gains into Eq. \ref{eqn:omega0} and Eq. \ref{eqn:zeta}.
\begin{equation}
\omega_0 = \frac{v}{R}
\left( 1 - (\cos \eta - \frac{\sin \eta}{kR}) \right)^{-1/2}
\label{eqn:path_omega0}
\end{equation}
%
\begin{equation}
\zeta = \frac{\sin \eta + \frac{\cos \eta}{kR}}
{2 \sqrt{1 - (\cos \eta - \frac{\sin \eta}{kR})}}
\label{eqn:path_zeta}
\end{equation}
%
These equations may be combined to eliminate $k$ as follows:
\begin{equation}
kR = \frac{\omega_0^2 \sin \eta}{v^2/R^2 - \omega_0^2 (1 - \cos \eta)}
\label{eqn:kr}
\end{equation}
%
\begin{equation}
(1 - \cos \eta) (\omega_0 R / v)^2 - 2 \zeta \sin \eta \cdot (\omega_0 R /v) +
\cos \eta = 0
\end{equation}
%
\begin{equation}
\omega_0 = \frac{v}{R} \frac{
\zeta \sin \eta - \sqrt{\zeta^2 \sin^2 \eta - \cos \eta (1 - \cos \eta)}
}{
1 - \cos \eta
}
\label{eqn:omega_zeta_eta}
\end{equation}

\begin{table}
\begin{center}
\begin{tabular}{ccccccc}
\hline
\hline
$v$ & $R$ & $\zeta$ & $\omega_0/2 \pi$ & $\eta$ & $t$ & $r_{\mathrm{min}}$ \\\
[m/s] & [m] & [\#] & [Hz] & [\#] & [s] & [m] \\
\hline
30 & 110 & 0.707 & 0.1 & 0.462 & 1.69 & 32 \\
60 & 110 & 0.616 & 0.117 & 0.924 & 1.69 & 32 \\
90 & 110 & 0.508 & 0.13 & 1.386 & 1.69 & 32 \\
\hline
30 & 110 & 0.918 & 0.075 & 0.364 & 1.33 & 130 \\
60 & 110 & 0.707 & 0.1 & 0.727 & 1.33 & 130 \\
90 & 110 & 0.562 & 0.119 & 1.09 & 1.33 & 130 \\
\hline
30 & 110 & 1.30 & 0.059 & 0.298 & 1.09 & 293 \\
60 & 110 & 0.926 & 0.08 & 0.596 & 1.09 & 293 \\
90 & 110 & 0.707 & 0.1 & 0.894 & 1.09 & 293 \\
\hline
\hline
\end{tabular}
\caption{Selecting path controller parameters with fixed $\zeta$ or
  look-ahead time ($t = \eta / (v/R)$).}
\label{tbl:choose_parameters}
\end{center}
\end{table}

Now the various tunable parameters ($r_{\mathrm{min}}$ and $\eta$) may
be chosen by first selecting a desired damping ratio $\zeta$ and
bandwidth.  Then, Eq. \ref{eqn:omega_zeta_eta} is used to select an
$\eta$ that meets the desired bandwidth.  Finally, these values are
converted to $k$ and thus $r_{\mathrm{min}}$ through Eq. \ref{eqn:kr}.
Examples of selecting parameters with this method are shown in
Tbl. \ref{tbl:choose_parameters}.

\begin{figure}[!ht]
\begin{center}
\begin{tikzpicture}
\begin{axis}[
    scale=1.3,
    xlabel=$\eta$,
    xmin=0,
    xmax=3,
    ymin=-1.3,
    ymax=4,
    grid=both
  ]
  \def\R{110};
  \def\rmin{75};
  \def\kR{(pi/2/\rmin * \R)};
  \def\z{(cos(deg(x)) - sin(deg(x)) / \kR)};
  \def\kp{(\z / (1 - \z))};
  \def\kd{((sin(deg(x)) + cos(deg(x)) / \kR) / (1 - \z))};
  \def\omegafunc{sqrt(1 + \kp)};
  \def\zetafunc{(\kd / (2 * \omegafunc))};
  \addplot [smooth, color=blue, mark=none, domain=0:3] {\kp};
  \addplot [smooth, color=green, mark=none, domain=0:3] {\kd};
  \addplot [smooth, color=red, mark=none, domain=0:3] {\omegafunc};
  \addplot [smooth, color=cyan, mark=none, domain=0:3] {\zetafunc};
  \addplot [smooth, color=black, domain=0:3] {0};
  %\addplot [smooth, color=gray, domain=0:3] {sqrt(2)/2};
  \legend{$k_p / (v/R)^2$, $k_d / (v/R)$, $\omega_0 / (v/R)$, $\zeta$};
\end{axis}
\end{tikzpicture}
\caption{Gains and parameters of the linearized curvature controller
  without the current curvature look-ahead ($R = 110, r_{\mathrm{min}} = 75$).
  Typically, $\eta \sim 1$ in normal flight conditions. All values are
  expressed in non-dimensional form.}
\label{fig:simple_gains}
\end{center}
\end{figure}


\section{Analysis of general path control}

In the general form of the path controller, the future position of the
kite is based on projecting the current measured curvature for a time
$t_1$ and then projecting a future commanded curvature for a time
$t_2$.  This is modeled by breaking up $\eta$ into two components:
%
\begin{equation}
\eta = \eta_1 + \eta_2
\end{equation}
%
and then applying Eq. \ref{eqn:delta_eta} and \ref{eqn:epsilon_eta}
twice.  The outputs of the first application of these equations are
the inputs to the next application.  Thus, the crosstrack offset and
heading offset at time $t_1$ is:
%
\begin{eqnarray}
\delta(\eta_1) &=& \cos \eta_1 \cdot \delta_0 + (1 - \cos \eta_1) \delta r_0 +
R \sin \eta_1 \cdot \epsilon_0 \\
\epsilon(\eta_1) &=& -\frac{\sin \eta_1}{R} \delta_0 +
\frac{\sin \eta_1}{R} \delta r_0 +
\cos \eta_1 \cdot \epsilon_0
\end{eqnarray}
%
And, the crosstrack offset and heading offset at time $t = t_1 + t_2$
is:
%
\begin{eqnarray}
\delta(\eta) &=& \cos \eta_2 \left(
\cos \eta_1 \cdot \delta_0 + (1 - \cos \eta_1) \delta r_0 +
R \sin \eta_1 \cdot \epsilon_0 \right) \\
&+& (1 - \cos \eta_2) \delta r \nonumber \\
&+& R \sin \eta_2 \left(-\frac{\sin \eta_1}{R} \delta_0 +
\frac{\sin \eta_1}{R} \delta r_0 +
\cos \eta_1 \cdot \epsilon_0 \right) \nonumber \\
\epsilon(\eta) &=& -\frac{\sin \eta_2}{R} \left(
\cos \eta_1 \cdot \delta_0 + (1 - \cos \eta_1) \delta r_0 +
R \sin \eta_1 \cdot \epsilon_0 \right) \\
&+& \frac{\sin \eta_2}{R} \delta r \nonumber \\
&+& \cos \eta_2 \left(-\frac{\sin \eta_1}{R} \delta_0 +
\frac{\sin \eta_1}{R} \delta r_0 +
\cos \eta_1 \cdot \epsilon_0 \right) \nonumber
\end{eqnarray}

From these equations, it is possible to extract the linearized form of
the curvature command:
%
\begin{alignat}{3}
\delta \kappa_{\mathrm{cmd}} &= -\frac{1}{R^2} \bigg[&\bigg(&
-k \cos \eta_2 \cos \eta_1 + k \sin \eta_2 \sin \eta_1 \\
& && + \sin \eta_2 \cos \eta_1/R + \cos \eta_2 \sin \eta_1/R
\bigg) \cdot \delta \nonumber \\
%
& &+& \bigg(-k \cos \eta_2 (1 - \cos \eta_1) -
k \sin \eta_2 \sin \eta_1 \nonumber \\
& && + \sin \eta_2 (1 - \cos \eta_1)/R -
\cos \eta_2 \sin \eta_1 / R \bigg) \cdot \delta r \nonumber \\
%
& &+& \bigg(-kR \cos \eta_2 \sin \eta_1 -
kR \sin \eta_2 \cos \eta_1 \nonumber \\
& && + \sin \eta_2 \sin \eta_1 - \cos \eta_2 \cos \eta_1
\bigg) \cdot \epsilon \bigg] \nonumber \\
%
& &\bigg/& \left( \frac{\sin \eta_2}{R} + k (1 - \cos \eta_2) \right) \nonumber
\end{alignat}
%
which simplifies to:
\begin{alignat}{3}
\delta \kappa_{\mathrm{cmd}} &= \frac{1}{R^2} \bigg[&\bigg(&
\cos \eta - \frac{\sin \eta}{kR} \bigg) \cdot \delta \\
%
& &+& \bigg(-\cos \eta + \cos \eta_2
+ \frac{\sin \eta}{kR} - \frac{\sin \eta_2}{kR} \bigg) \cdot \delta r \nonumber \\
%
& &+& \bigg(\sin \eta + \frac{\cos \eta}{kR} \bigg) \cdot R \epsilon \bigg] \nonumber \\
%
& &\bigg/& \left( 1 - (\cos \eta_2 - \frac{\sin \eta_2}{kR}) \right) \nonumber
\end{alignat}
%

\begin{figure}[!ht]
\begin{center}
\begin{tikzpicture}
\begin{axis}[
    scale=1.3,
    xlabel=$\eta$,
    xmin=0,
    xmax=3,
    ymin=-1.3,
    ymax=4,
    grid=both
  ]
  \def\R{110};
  \def\rmin{75};
  \def\kR{(pi/2/\rmin * \R)};
  \def\neta{(x)};
  \def\netaone{(0.2 * x)};
  \def\netatwo{(0.8 * x)};
  \def\den{(1 - (cos(deg(\netatwo)) - sin(deg(\netatwo)) / \kR))};
  \def\kp{((cos(deg(\neta)) - sin(deg(\neta)) / \kR) / \den)};
  \def\kd{((sin(deg(\neta)) + cos(deg(\neta)) / \kR) / \den)};
  \def\ka{((-cos(deg(\neta)) + sin(deg(\neta)) / \kR +
            cos(deg(\netatwo)) - sin(deg(\netatwo)) / \kR) /
            \den)};
  \def\omegafunc{sqrt(1 + \kp / (1 + \ka))}
  \def\omegaorig{sqrt(1 + \kporig)}
  \def\zetafunc{((\kd / (1 + \ka)) / (2 * \omegafunc))};
  \addplot [smooth, color=blue, mark=none, domain=0:3] {\kp};
  \addplot [smooth, color=green, mark=none, domain=0:3] {\kd};
  \addplot [smooth, color=yellow, mark=none, domain=0:3] {\ka};
  \addplot [smooth, color=red, mark=none, domain=0:3] {\omegafunc};
  \addplot [smooth, color=cyan, mark=none, domain=0:3] {\zetafunc};
  \addplot [smooth, color=black, domain=0:3] {0};
  %\addplot [smooth, color=gray, domain=0:3] {sqrt(2)/2};
  \legend{$k_p / (v/R)^2$, $k_d / (v/R)$, $k_a$, $\omega_0 / (v/R)$, $\zeta$};
\end{axis}
\end{tikzpicture}
\caption{Gains and parameters of the linearized curvature controller
  {\it with} the current curvature look-ahead
  ($R = 110, r_{\mathrm{min}} = 75, \eta_1 = \eta/5$).
  All values are expressed in non-dimensional form.}
\label{fig:general_gains}
\end{center}
\end{figure}
%
In addition to the PD terms, the curvature command also has an
acceleration (i.e. curvature) term:
%
\begin{equation}
v^2 \delta \kappa_{\mathrm{cmd}} =
k_p \delta + k_d v \epsilon - k_a v^2 \delta \kappa
\label{eqn:crosstrack_with_accel}
\end{equation}
%
The proportional, derivative, and acceleration gains are:
%
\begin{eqnarray}
k_p &=& \frac{v^2}{R^2} \frac{
\cos \eta - \frac{\sin \eta}{k R}}
{1 - (\cos \eta_2 - \frac{\sin \eta_2}{k R})} \\
%
k_d &=& \frac{v}{R} \frac{\sin \eta + \frac{\cos \eta}{kR}}
{1 - (\cos \eta_2 - \frac{\sin \eta_2}{kR})} \nonumber \\
k_a &=& \frac{-\cos \eta + \frac{\sin \eta}{kR} + \cos \eta_2 -
  \frac{\sin \eta_2}{kR}}
{1 - (\cos \eta_2 - \frac{\sin \eta_2}{kR})}
\end{eqnarray}

The inner loop, which controls tether roll to meet a commanded
curvature, may be modeled as a simple low-pass filter with cut-off
frequency $\omega_{\kappa}$.
\begin{equation}
  \delta \dot{\kappa} = \omega_{\kappa}
  (\delta \kappa_{\mathrm{cmd}} - \delta \kappa)
  \label{eqn:inner_loop_model}
\end{equation}
%
Then, from combining Eq. \ref{eqn:delta_diffeq},
\ref{eqn:crosstrack_with_accel}, and \ref{eqn:inner_loop_model}, the
closed loop dynamics of the system are:
\begin{equation}
  \dddot{\delta} + (k_d + \omega_{\kappa} (1 + k_a)) \ddot{\delta} +
  (v^2/R^2 + \omega_{\kappa} k_d + k_p) \dot{\delta} +
  \omega_{\kappa} (v^2/R^2 (1 + k_a) + k_p) \delta = 0
\end{equation}
%
It is useful to examine the $\omega_{\kappa} \gg 1$ case to compare
this system with the basic path control case.
\begin{equation}
  \ddot{\delta} + \frac{k_d}{1 + k_a} \dot{\delta} +
  \left(\frac{v^2}{R^2} + \frac{k_p}{1 + k_a} \right) \delta = 0
\end{equation}
%
It is clear that the proportional and derivative gains must be
increased to account for the acceleration gain.  However, $\omega$ and
$\zeta$ still only depend on $\eta$ and are given by the same formulas
as the basic case (Eq. \ref{eqn:path_omega0} and
Eq. \ref{eqn:path_zeta}).  Note that the acceleration gain cannot be
increased to the point where $\eta = \eta_1$ or the controller will
run into issues with saturations and other nonlinearities due to the
high proportional and derivative gains.

The acceleration gain does two things.  First, it increases the
disturbance rejection on the curvature inner loop.  This may be seen
by separating the acceleration gain from the proportional and
derivative gains as shown in Fig. \ref{fig:path_inner_outer}.
Increasing the disturbance rejection in the manner is important
because the inner loop actually controls tether roll, which is related
to but is not identical to curvature.  Second, the acceleration gain
makes the nonlinear, anticipatory aspects of the curvature control
more accurate.


\begin{figure}[!ht]
\begin{center}
\begin{tikzpicture}
  \matrix [row sep=4mm, column sep=8mm] {
    \node[dspnodeopen, dsp/label=above] (n00) {$\delta_{\mathrm{cmd}}$}; &[-4mm]
    \node[vdspadder] (n01) {}; &[-4mm]
    \node[dspfilter] (n02) {$C_{\mathrm{path}}$}; &[-8mm]
    \node[vdspadder] (n03) {}; &[4mm]
    \node[dspfilter] (n05) {$\kappa \rightarrow \phi_t$}; &[-4mm]
    \node[dspfilter] (n06) {$G^{\mathrm{closed}}_{\mathrm{lateral}}$}; &[-4mm]
    \node[dspfilter] (n07) {$\phi_t \rightarrow \kappa$}; &
    \node[dspnodefull] (n08) {}; &[-4mm]
    \node[dspfilter] (n09) {$G_{\mathrm{path}}$}; &[-4mm]
    \node[dspnodefull] (n0a) {}; &
    \node[dspnodeopen, dsp/label=above] (n0b) {$\delta$}; \\
    \node[coordinate] (n10) {}; &
    \node[coordinate] (n11) {}; &
    \node[coordinate] (n12) {}; &
    \node[coordinate] (n13) {}; &
    \node[coordinate] (n15) {}; &
    \node[dspsquare] (n16) {$k_a$}; &
    \node[coordinate] (n17) {}; &
    \node[coordinate] (n18) {}; &
    \node[coordinate] (n19) {}; &
    \node[coordinate] (n1a) {}; &
    \node[coordinate] (n1b) {}; & \\
    \node[coordinate] (n20) {}; &
    \node[coordinate] (n21) {}; &
    \node[coordinate] (n22) {}; &
    \node[coordinate] (n23) {}; &
    \node[coordinate] (n25) {}; &
    \node[coordinate] (n26) {}; &
    \node[coordinate] (n27) {}; &
    \node[coordinate] (n28) {}; &
    \node[coordinate] (n29) {}; &
    \node[coordinate] (n2a) {}; &
    \node[coordinate] (n2b) {}; & \\
  };

  \draw[dspconn] (n00) -- (n01);
  \draw[dspconn] (n01) -- (n02) -- (n03);
  \draw[dspconn] (n03) -- node[above] {$\delta \kappa_{\cmd}$} (n05)
                       -- (n06) -- (n07)
                       -- node[above] {$\delta \kappa$} (n08)
                       -- (n09) -- (n0a) -- (n0b);
  \draw[dspconn] (n08) |- (n16) -| (n03);
  \draw[dspconn] (n0a) |- (n2a) -| (n01);
\end{tikzpicture}
\caption{Schematic of the path control loop.}
\label{fig:path_inner_outer}
\end{center}
\end{figure}

\section{Final notes}

This analysis shows that the nonlinear path controller is equivalent
to a more traditional linear crosstrack controller for small
crosstrack and heading offsets.  And also, that the parameters
($\zeta$ and $\omega_0$) of the equivalent linear crosstrack
controller may be set using the parameters of the nonlinear path
controller (i.e. $\eta$ and $r_{\mathrm{min}}$).

However, it remains to be shown whether the nonlinear path controller
actually offers any advantages over the simple linear crosstrack
controller or over other nonlinear path tracking schemes
\cite{park2004}.

\begin{thebibliography}{1}
\bibitem{park2004} Park, Sanghyuk, John Deyst, and Jonathan
  P. How. ``A new nonlinear guidance logic for trajectory tracking.''
  {\it Proceedings of the AIAA Guidance, Navigation and Control
    Conference}. 2004.
\end{thebibliography}

\end{document}
