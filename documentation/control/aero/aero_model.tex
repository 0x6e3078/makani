\documentclass[11pt]{amsart}
\usepackage{geometry}
\geometry{letterpaper}
\usepackage{graphicx}
\usepackage{amssymb}
\usepackage{amsmath}
\usepackage{epstopdf}
\usepackage{fancyhdr}

\begin{document}

%% \section{Aerodynamics model}

%% \subsection{Description of problem}

%% To model the dynamics of the system, the simulator needs to know the
%% aerodynamic force and moment on the wing.  To make the problem
%% tractable, we only model {\it steady} aerodynamic forces and moments.
%% However, the forces and moments on the wing are still a function of
%% the airspeed, $v$, angle-of-attack, $\alpha$, sideslip, $\beta$,
%% angular rate, $\vec{\omega}$, flap deflections, $\delta_i$, and
%% propeller angular rates $\Omega_i$:
%% \begin{equation}
%% \label{eqn:force-moment-function}
%% \left(
%% \begin{array}{c}
%% \vec{f}_b \\
%% \vec{m}_b
%% \end{array}
%% \right) = \vec{F}(v, \alpha, \beta, \vec{\omega}, \delta_1, ..., \delta_8,
%% \Omega_1, ..., \Omega_8)
%% \end{equation}
%% Because of the large number of dimensions, it would be impractical
%% generate a look-up table for this very general function.  Fortunately,
%% we can make a few simplifications.  During crosswind flight, {\it
%%   i.e.} airplane-like flight, there is little interaction between
%% rotor wake and the airframe, and thus we can reasonably separate the
%% airframe and motor force-moments:
%% \begin{equation}
%% \label{eqn:force-moment-separate}
%% \left(
%% \begin{array}{c}
%% \vec{f}_b \\
%% \vec{m}_b
%% \end{array}
%% \right) \approx \vec{F}_{\mathrm{airframe}}(v, \alpha, \beta, \vec{\omega},
%% \delta_1, ..., \delta_8) +
%% \vec{F}_{\mathrm{prop}}(v, \alpha, \beta, \vec{\omega}, \Omega_1, ..., \Omega_8)
%% \end{equation}
%% Also, if we assume the forces on the airframe scale as $v^2$, then the
%% airframe force-moment can be expressed in terms of coefficients.  This
%% eliminates $v$ from the look-up table:
%% \begin{equation}
%% \label{eqn:force-moment-coefficient}
%% \vec{F}_{\mathrm{airframe}}(v, \alpha, \beta, \vec{\omega},
%% \delta_1, ..., \delta_8) \approx \frac{1}{2} \rho v^2 A
%% \cdot
%% C_{\vec{F}}(\alpha, \beta, \vec{\omega}, \delta_1, ..., \delta_8)
%% \cdot
%% [1, 1, 1, b, c, b]^T
%% \end{equation}
%% Finally, we can linearize the total coefficient function about a
%% nominal $\hat{\omega}_0$ and $\vec{\delta}_0$ as follows:
%% \begin{equation}
%% \label{eqn:force-moment-derivative}
%% C_{\vec{F}}(\alpha, \beta, \vec{\omega}, \delta_1, ..., \delta_8)
%% =
%% C^0_{\vec{F}}(\alpha, \beta) +
%% \frac{\partial C_{\vec{F}}}{\partial \hat{\omega}}(\alpha, \beta)
%% \cdot (\hat{\omega} - \hat{\omega}_0) +
%% \frac{\partial C_{\vec{F}}}{\partial \vec{\delta}}(\alpha, \beta)
%% \cdot (\vec{\delta} - \vec{\delta}_0)
%% \end{equation}
%% (A more common notation for these coefficients is shown in Table
%% \ref{tbl:coefficients}.)  Now we only need a look-up table on $\alpha$
%% and $\beta$, which is very managable.

%% \begin{center}
%%   \begin{table}[h]
%%     \label{tbl:coefficients}
%%     \caption{Symbols for the various coefficients, stability
%%       derivatives, and control derivatives.}
%%     \begin{tabular}{lllllll}
%%       \hline
%%       \hline
%%       Type & \multicolumn{6}{c}{Symbol} \\
%%       \hline
%%       Coefficient
%%       & $C_x$ & $C_y$ & $C_z$ & $C_l$ & $C_m$ & $C_n$ \\
%%       Stability derivatives
%%       & $C_{x_p}$ & $C_{y_p}$ & $C_{z_p}$ & $C_{l_p}$ & $C_{m_p}$ & $C_{n_p}$ \\
%%       & $C_{x_q}$ & $C_{y_q}$ & $C_{z_q}$ & $C_{l_q}$ & $C_{m_q}$ & $C_{n_q}$ \\
%%       & $C_{x_r}$ & $C_{y_r}$ & $C_{z_r}$ & $C_{l_r}$ & $C_{m_r}$ & $C_{n_r}$ \\
%%       Control derivatives
%%       & $C_{x_{\delta_i}}$ & $C_{y_{\delta_i}}$ & $C_{z_{\delta_i}}$
%%       & $C_{l_{\delta_i}}$ & $C_{m_{\delta_i}}$ & $C_{n_{\delta_i}}$ \\
%%       & $C_{x_{\Omega_i}}$ & $C_{y_{\Omega_i}}$ & $C_{z_{\Omega_i}}$
%%       & $C_{l_{\Omega_i}}$ & $C_{m_{\Omega_i}}$ & $C_{n_{\Omega_i}}$ \\
%%       \hline
%%       \hline
%%     \end{tabular}
%%   \end{table}
%% \end{center}

%% \begin{equation}
%% \vec{\Omega}_i = [
%% \Omega^0_{\mathrm{bot}},\; \Omega^0_{\mathrm{bot}},\; \Omega^0_{\mathrm{bot}},\;
%% \Omega^0_{\mathrm{bot}},\; \Omega^0_{\mathrm{top}},\; \Omega^0_{\mathrm{top}},\;
%% \Omega^0_{\mathrm{top}},\; \Omega^0_{\mathrm{top}}]^T
%% \end{equation}

%% \begin{equation}
%% N \sim N_{\alpha} \times N_{\beta} \times
%% \left(1 + N_{\omega\;\mathrm{deriv.}} + N_{\delta\;\mathrm{deriv.}}\right) \sim 420
%% \end{equation}



%% \begin{equation}
%% N \sim N_v \times N_{\alpha} \times N_{\beta} \times \sum_i N_{\delta_i} \times
%% N_{\omega\;\mathrm{deriv.}} \sim 83000
%% \end{equation}


%% \begin{equation}
%% T(\Omega, v_x)
%% \end{equation}
%% \begin{equation}
%% \tau(\Omega, v_x)
%% \end{equation}

%% \begin{equation}
%% \label{eqn:force-moment-separate}
%% \left(
%% \begin{array}{c}
%% \vec{f}_b \\
%% \vec{m}_b
%% \end{array}
%% \right) \approx \vec{F}^0(v, \alpha, \beta, \vec{\omega}) +
%% \sum_i \Delta \vec{F}^i(v, \alpha, \beta, \vec{\omega}, \delta_i)
%% \end{equation}


%% So far, this is all textbook aircraft dynamics modeling.  The unique
%% aspects of our problem occur during hover and the transition to
%% crosswind flight.  During these phases, the rotor wake interacts very
%% strongly with the elevator.  Obviously, this eliminates the
%% airframe-motor separation simplification in
%% Eq. \ref{eqn:force-moment-separate}.  However, it also eliminates the
%% coefficient simplification in Eq. \ref{eqn:force-moment-coefficient}
%% because the forces now depend on the rotor wake speed, which is mostly
%% independent of the airspeed (especially at the low airspeeds
%% encountered during hovering and transitioning flight).  Our problem is
%% to figure out the best way of modeling this system without requiring a
%% huge look-up table.  Additionally, for the nominal operating points
%% where we expect the wing to fly, we want to model the system with
%% relatively high fidelity methods, {\it e.g.}  Reynolds averaged
%% Navier-Stokes.  While far away from the nominal operating range, we
%% still want our aerodynamics model to return reasonable values to the
%% simulator; though we can accept a significant decrease in fidelity.


%% \subsection{Direct approach}

%% The most direct approach to modeling the aerodynamic force-moment
%% function, Eq. \ref{eqn:force-moment-function}, is to run aerodynamic
%% simulations at each point in a table of $(v, \alpha, \beta, ...)$
%% values.  By judiciously selecting the range of conditions for each
%% flight regime, we can keep the number of operating points relatively
%% low.  Table \ref{tbl:operating-point-table} shows the recommended
%% operating points for each flight regime.

%% \begin{center}
%%   \begin{table}[h]
%%     \caption{Recommended operating points for each flight regime.}
%%     \label{tbl:operating-point-table}
%%     \begin{tabular}{llll}
%%       \hline
%%       \hline
%%       Flight regime & Parameters & Values & Units \\
%%       \hline
%%       Hover
%%       & $v$            & [2.5, 4, 6.3, 10, 16, 25]              & m/s \\
%%       & $\alpha$       & [60, 90, 120]                          & deg \\
%%       & $\beta$        & [0, 30, 60, 90]                        & deg \\
%%       & $\hat{\omega}$ & $\left<0, 0, 0\right>$                 & \# \\
%%       & $\delta_i$     & $\vec{\delta}_{\mathrm{hover}}(v, \alpha)$ & deg \\
%%       & $T_i$          & $\vec{T}_{\mathrm{hover}}$                 & N \\

%%       \hline
%%       Transition
%%       & $v$            & [2.5, 4, 6.3, 10, 16, 25]                 & m/s \\
%%       & $\alpha$       & [-5, 0, 5, 10, 30, 60, 90]                & deg \\
%%       & $\beta$        & [-5, 0, 5]                                & deg \\
%%       & $\hat{\omega}$ & $\left<0, 0, 0\right>$                    & \# \\
%%       & $\delta_i$     & $\vec{\delta}_{\mathrm{hover}}(v, \alpha)$    & deg \\
%%       & $T_i$          & $\vec{T}_{\mathrm{thrust}}(v, \alpha, \beta)$ & N \\

%%       \hline
%%       Crosswind
%%       & $v$            & [30, 60, 90]                              & m/s \\
%%       & $\alpha$       & [-5, 0, 5]                                & deg \\
%%       & $\beta$        & [-5, 0, 5]                                & deg \\
%%       & $\hat{\omega}$ & [$\left<0, 0, 0\right>$,
%%                           $\left<0, 0.0022, -0.1653\right>$]       & \# \\
%%       & $\delta_i$     & $\vec{\delta}_{\mathrm{crosswind}}(\alpha)$    & deg \\
%%       & $T_i$          & $\left<0, 0, 0, 0, 0, 0, 0, 0\right>$      & N \\
%%       \hline
%%       \hline
%%     \end{tabular}
%%   \end{table}
%% \end{center}

%% In the hover regime, the wing is generally very close to $\alpha =
%% 90^{\circ}$ and $\beta = 0^{\circ}$.  Multiple angles-of-attack are
%% used to capture the changes on either side of this orientation.
%% Multiple sideslip angles are used to model the wing being disturbed by
%% gusts from the side.  It is assumed that the forces are symmetric
%% about $\pm \beta$, so only the positive sideslip angles are
%% calculated.  The number of hover runs is: $6 \times 3 \times 4 = 72$.

%% In the transition regime, the wing passes through a wide range of
%% angles-of-attack.  And, like the hover case, we cannot remove $v$ from
%% the look-up table because the elevator interacts strongly with both
%% the apparent wind and propwash. The number of transition runs is: $6
%% \times 7 \times 3 = 126$.

%% In the crosswind regime, the coefficients should mostly be independent
%% of apparent wind speed because there is very little interaction
%% between the rotor wake and elevator.  However, multiple airspeeds are
%% still included in the runs to capture Reynolds number effects.  The
%% operating range of angles-of-attack and sideslips are relatively
%% small.  Pylon and wing stall will be modeled separately, and thus CFD
%% will only be used to determine coefficents in the unstalled regime.
%% Two nominal angular rates are used to capture the dynamics during
%% straight flight, for example, near the end of transition, and normal
%% crosswind flight.  The number of crosswind runs is: $3 \times 3 \times
%% 3 \times 2 = 54$.

%% Note that flap deflections, $\delta_i$, and thrusts, $T_i$, may be a
%% function of other parameters.  For example, in hover the elevator
%% should be closely aligned with the vector sum of the apparent wind and
%% propwash.

%% All together, we have identified 252 operating points where we'd like
%% to generate a high fidelity model.  Moreover, it may be necessary to
%% multiply this by a factor of 12 to account for the stability and flap
%% derivatives (1 nominal, 3 angular rates, 8 flaps).

%% \subsection{Component approach}

%% To avoid the explosion in operating points from a high-dimensional
%% database, it may be necessary to separate the coefficient analysis
%% into a main wing component that includes the wing, ailerons, pylons,
%% and fuselage, and a tail component that includes the horizontal and
%% vertical stabilizers and rudder.  (Because the entire horizontal
%% stabilizer rotates, the elevator {\it is} the horizontal stabilizer.)
%% All coefficients are referenced to the same area, span, and chord and
%% are calculated about the body center.  Using this strategy, the total
%% coefficients may be calculated as the sum of the coefficients from
%% each component, accounting for the fact that the tail experiences a
%% different dynamic pressure from the propeller slipstream and different
%% incidence angles from the propeller slipstream and wing downwash.
%% \begin{equation}
%% C_i(\bar{q}, \alpha, \beta) = {C_i}^{\mathrm{main}}(\alpha, \beta)
%% + \frac{\bar{q}_t}{\bar{q}} {C_i}^{\mathrm{tail}}(\alpha_t, \beta_t)
%% \end{equation}
%% Here $i$ references any of the standard coefficient or derivative
%% subscripts (see Tbl. \ref{tbl:coefficients}).  Note that each
%% coefficient is simply a lookup table with two variables, and thus does
%% not require evaluation at a huge number of operating points.

%% This method, however, shifts the high-dimensional aspects of the
%% problem into calculating the local dynamic pressure, $\bar{q}_t$, and
%% incidence angles, $\alpha_t$ and $\beta_t$, at the tail.  These values
%% are a function of the rotor thrust, $T$, the apparent wind speed, $v$,
%% and incidence angles, $\alpha$ and $\beta$, of the body as a whole.
%% \begin{eqnarray}
%% \left(
%% \begin{array}{c}
%% \bar{q}_t \\
%% \alpha_t \\
%% \beta_t
%% \end{array}
%% \right) = G(T, v, \alpha, \beta)
%% \end{eqnarray}
%% We cannot remove the $v$ dimension from this function by just
%% calculating the non-dimensional coefficients because the incidence
%% angles at the tail will be determined by the vector sum of the
%% apparent wind and propeller slipstream.  For example, during hover
%% when $\alpha = 90^{\circ}$ and $\beta = 0^{\circ}$, the incidence
%% angle at the horizontal stabilizer will be approximately $\tan^{-1}(v
%% / v_{\mathrm{wake}})$.  It is however, relatively straightforward to
%% calculate this function using either analytical methods or fast
%% aerodynamic codes such as \textsc{vsaero}.

\section{Description of database formats}

We use two different aerodynamic database formats depending on the
flight regime.  Both databases are represented in terms of
dimensionless force-moment coefficients, $C_x$, $C_y$, $C_z$, $C_l$, $C_m$,
and $C_n$, rather than the forces and moments themselves.  These
coefficients are converted to the forces and moments as follows:
\begin{equation}
\label{eqn:force-moment-coefficient}
\left(
\begin{array}{c}
\vec{f}_b \\
\vec{m}_b
\end{array}
\right) = \frac{1}{2} \rho v^2 A
\cdot
C_{\vec{F}}(\alpha, \beta, \vec{\omega}, \delta_1, ..., \delta_8)
\cdot
[1, 1, 1, b, c, b]^T
\end{equation}
Here $\vec{F}$ indicates the subscripts for the force and moment
coefficients: $x, y, z, l, m, n$.

For crosswind flight, {\it i.e.} unstalled flight, we
have a look-up table on $\alpha$ and $\beta$ of the nominal
force-moment coefficients $C_{\vec{F}}^0$, stability derivatives
$\partial C_{\vec{F}}/\partial \hat{\omega}$, and control
derivatives $\partial C_{\vec{F}}/\partial \vec{\delta}$.
These are combined into the total force-moment coefficients as
follows:
\begin{equation}
\label{eqn:force-moment-derivative}
C_{\vec{F}}(\alpha, \beta, \vec{\omega}, \delta_1, ..., \delta_8)
=
C^0_{\vec{F}}(\alpha, \beta) +
\frac{\partial C_{\vec{F}}}{\partial \hat{\omega}}(\alpha, \beta)
\cdot (\hat{\omega} - \hat{\omega}_0) +
\frac{\partial C_{\vec{F}}}{\partial \vec{\delta}}(\alpha, \beta)
\cdot (\vec{\delta} - \vec{\delta}_0)
\end{equation}
where $\hat{\omega}_0$ and $\vec{\delta}_0$ are the nominal angular
rate and flap deflections.  Note that $\hat{\omega}$ is the {\it
  reduced} angular rate, which is defined as:
\begin{equation}
  \hat{\omega} = \vec{\omega} \cdot
\left<\frac{b}{2 v}, \frac{c}{2 v}, \frac{b}{2 v}\right>
\end{equation}

For the hover and transition aerodynamic databases, we add airspeed
and flap deflection as dimensions of the look-up table.  Airspeed is
added because the rotor wake is advected with the wind during hover.
Flap deflection is added so we can deflect some flaps, specifically
the elevator, beyond their linear range.  To reduce dimensions, we
assume the flaps are weakly interacting and thus only need to perform
a look-up on a single flap deflection at a time.  Thus, we have a
nominal force-moment coefficient $C^0_{\vec{F}}$, increments to this
coefficient from each flap deflection $\Delta C^i_{\vec{F}}$, and the
stability derivatives for each of these components, $\partial
C^0_{\vec{F}}/\partial \hat{\omega}$ and $\partial \Delta
C^i_{\vec{F}}/\partial \hat{\omega}$.
\begin{eqnarray}
\label{eqn:dvl-model}
C_{\vec{F}}(v, \alpha, \beta, \vec{\omega}, \delta_1, ..., \delta_8) &=&
C^0_{\vec{F}}(v, \alpha, \beta) +
\sum_i \Delta C^i_{\vec{F}}(v, \alpha, \beta, \delta_i) \\
&+& \left(
\frac{\partial C^0_{\vec{F}}}{\partial \hat{\omega}}(v, \alpha, \beta) +
\sum_i \frac{\partial \Delta C^i_{\vec{F}}}{\partial \hat{\omega}}(v, \alpha, \beta, \delta_i)
\right) \cdot \hat{\omega}
\end{eqnarray}

The {\it direct} forces and moments from the propellers are added in
separately.  However, the {\it indirect} effects from the propellers
such as the interaction of the propwash with the empennage are
included in these coefficients.


%% \section{Methods}

%% \begin{center}
%%   \begin{table}[h]
%%     \label{tbl:airframe_methods}
%%     \caption{}
%%     \begin{tabular}{lllllll}
%%       \hline
%%       \hline
%%       Method & Examples & Run time [s] \\
%%       \hline
%%       Analytical & Flat plate & \< 1 \\
%%       Vortex lattice & \textsc{AVL}, \textsc{ASWING} & \< 1 \\
%%       Panel methods & \textsc{PANAIR}, \textsc{VSAERO} & 10 - 1000 \\
%%       ??? \\
%%       Full Navier-Stokes CFD & Star-CCM+ & 10000 - 50000 \\
%%       \hline
%%       \hline
%%     \end{tabular}
%%   \end{table}
%% \end{center}



%% \section{Hover model}

%% \subsection{Analytical calculations}

%% \newcommand{\propwash}{\mathrm{p.w.}}

%% \begin{equation}
%% m_y(q) = -\frac{1}{2} \rho v_{pw}^2 A_{ht} l_{ht}
%% \left( 2 \pi \frac{\AR_{ht}}{2 + \AR_{ht}} \frac{q l_{ht}}{v_{pw}} \right)
%% \end{equation}

%% \begin{equation}
%% C_{m_q} = \pi \frac{\AR_{ht}}{2 + \AR_{ht}} \frac{v_{pw} A_{ht} l_{ht}^2}{v A_w c^2}
%% \end{equation}


%% \begin{equation}
%% C_{n_r} = \pi \frac{\AR_{vt}}{2 + \AR_{vt}} \frac{v_{pw} A_{vt} l_{vt}^2}{v A_w b^2}
%% \end{equation}


\end{document}
