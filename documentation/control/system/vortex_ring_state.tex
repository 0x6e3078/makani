\documentclass[11pt]{amsart}
\usepackage{geometry}
\geometry{letterpaper}
\usepackage{graphicx}
\usepackage{amssymb}
\usepackage{amsmath}
\usepackage{epstopdf}
\usepackage{fancyhdr}
\usepackage{tikz}
\usepackage{caption}

\title{Estimating propeller thrust and power during the vortex ring state}
\author{Makani Technologies LLC}

\begin{document}
\maketitle

\section{Definitions}
\begin{eqnarray*}
  C_T &\equiv& \text{Thrust coefficient} \\
  T &\equiv& \text{Thrust} \\
  P &\equiv& \text{Power} \\
  \rho &\equiv& \text{Density of air} \\
  A &\equiv& \text{Rotor disk area} \\
  R &\equiv& \text{Rotor radius} \\
  \sigma &\equiv& \text{Rotor solidity (i.e. blade area / disk area)} \\
  \theta &\equiv& \text{Rotor pitch} \\
  a &\equiv& \text{Two-dimensional lift slope factor $\sim 5.7$} \\
  V_c &\equiv& \text{Climb velocity (i.e. freestream velocity)} \\
  v_i &\equiv& \text{Induced velocity} \\
  v_h &\equiv& \text{Induced velocity during hover (i.e. when $V_c = 0$)} \\
  \Omega &\equiv& \text{Angular velocity [rad/s]} \\
  \lambda &\equiv& \text{Inflow factor} \\
\end{eqnarray*}

% Helicopter Flight Dynamics pg. 19 has a good picture of vortex ring, etc...

\section{Calculating the ``ideal'' thrust and power}

Momentum theory gives the formulas for thrust and power as a function
of the induced velocity:
\begin{equation}
  T = 2 \rho A (V_c + v_i) v_i
  \label{eqn:momentum_theory_thrust}
\end{equation}
\begin{equation}
  P = 2 \rho A (V_c + v_i)^2 v_i
\end{equation}
However, to calculate the induced velocity, we need some geometric
information about the propeller.  After all, different propellers
produce different thrusts at the same angular rate and climb rates.
\textit{Basic Helicopter Aerodynamics} gives a formula (pg.27, Eq. 3.22)
for the thrust coefficient using blade element theory assuming uniform
induced velocity across the rotor disk
\begin{equation}
  C_T = \frac{1}{2} \sigma a \left( \frac{1}{3} \theta - \frac{1}{2}
  \lambda \right)
\end{equation}
where
\begin{equation}
  \lambda = \frac{V_c + v_i}{\Omega R}.
\end{equation}
The total thrust is related to the thrust coefficient by the normal
relationship:
\begin{equation}
  T = \frac{1}{2} \rho (\Omega R)^2 C_T A
  \label{eqn:bet_thrust}
\end{equation}


We solve for the induced velocity $v_i$ by combining
Eq. \ref{eqn:momentum_theory_thrust} and Eq. \ref{eqn:bet_thrust}.
\begin{eqnarray*}
  2 \rho A (V_c + v_i) v_i  \frac{1}{2} \rho (\Omega R)^2 C_T A \\
  2 (V_c + v_i) v_i = \frac{1}{2} \Omega^2 R^2 \cdot \frac{1}{2} \sigma a
  \left( \frac{1}{3} \theta - \frac{1}{2} \lambda \right) \\
  = \frac{1}{12} \Omega^2 R^2 \sigma a \theta - \frac{1}{8} \Omega R \sigma a (V_c + v_i)
\end{eqnarray*}
Rearranging the terms:
\begin{equation}
  2 v_i^2 + \left(2 V_c + \frac{1}{8} \Omega R \sigma a\right) v_i +
  \frac{1}{8} \Omega R \sigma a V_c - \frac{1}{12} \Omega^2 R^2 \sigma a \theta
  = 0
\end{equation}
Solving the quadratic equation:
\begin{equation}
  v_i = -\left( \frac{1}{2} V_c + \frac{1}{32} \Omega R \sigma a \right)
  + \sqrt{ \left( \frac{1}{2} V_c - \frac{1}{32} \Omega R \sigma a \right)^2
    + \frac{1}{24} \Omega^2 R^2 \sigma a \theta}
\end{equation}
Now, it is simple to substitute this result back into the momentum
theory equations for thrust and power.  The variables $R$, $\sigma$,
$a$, and $\theta$ are all essentially geometric properties of the
propeller that do not change with different climb rates or angular
rates.  Also, these variables can be directly measured from the
propeller or estimated from the \textsc{xrotor} database.

\section{Modifying the result to account for vortex ring state}

The previous section demonstrates a method for extrapolating thrust
and power estimates beyond the range where \textsc{xrotor} provides
results.  However, it still uses the momentum theory approximation,
which is an idealization that breaks down in the vortex ring state.

One potential way of incorporating vortex ring state into our thrust
and power database is to assume a fudge factor, which is a function of
$V_c/v_h$, that multiplies the momentum theory thrust estimate.  This
multiplier, $k(V_c/v_h)$, is 1 when momentum theory applies and is
reduced during vortex ring state.
\begin{equation}
  T = k(V_c/v_h) \cdot 2 \rho A (V_c + v_i) v_i
  \label{eqn:mod_momentum_theory_thrust}
\end{equation}
The fudge factor accounts for the lost momentum that gets transferred
to the vortex rings rather an upward force.  The momentum theory power
formula is modified in a similar manner because power will still be
the product of thrust and the velocity of the airflow through the
actuator disk.
\begin{equation}
  P = k(V_c/v_h) \cdot 2 \rho A (V_c + v_i)^2 v_i
\end{equation}

The fudge factor may be estimated from Fig. 2.9 in \textit{Basic
  Helicopter Aerodynamics}, which shows a plot of $v_i/v_h$ versus
$V_c/v_h$ for a thrust equivalent to the hover thrust:
\begin{equation}
  k(V_c/v_h) = \frac{1}{(V_c/v_h + v_i/v_h) v_i/v_h}
\end{equation}
where $V_c/v_h$ and $v_i/v_h$ may just be read of the plot.  In
particular, for $V_c/v_h > 0$, $k(V_c/v_h) \approx 1$.  During vortex
ring state when $V_c/v_h = -1$, $k(-1) \approx 0.5$.  For the level of
accuracy that require, it is reasonable to assume that $k$ varies
linearly between $k(0)$ and $k(-1)$.

Now the argument from the previous section can be repeated with the
modified momentum theory equation,
Eq. \ref{eqn:mod_momentum_theory_thrust}.  The induced velocity is
calculated by solving the quadratic equation:
\begin{equation}
  2 k v_i^2 + \left(2 k V_c + \frac{1}{8} \Omega R \sigma a\right) v_i +
  \frac{1}{8} \Omega R \sigma a V_c - \frac{1}{12} \Omega^2 R^2 \sigma a \theta
  = 0
\end{equation}


\end{document}
