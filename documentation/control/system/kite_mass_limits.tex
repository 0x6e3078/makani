\documentclass[11pt]{amsart}

\usepackage{amsmath}
\usepackage{amssymb}
\usepackage{tikz}
\usepackage{pgfplots}
\usepackage{xcolor}

\usetikzlibrary{
  arrows,
  calc,
  decorations.markings,
  decorations.pathreplacing,
  dsp,
  fit,
  positioning
}
\newcommand\DrawAirfoil[1][]{
\begin{scope}[#1]
  % NACA 0012
  \draw plot [smooth] coordinates {
    (1.000000, 0.001260000)
    (0.9916796, 0.002421450)
    (0.9803692, 0.003981369)
    (0.9672663, 0.005761861)
    (0.9527169, 0.007706239)
    (0.9371981, 0.009743313)
    (0.9211172, 0.01181512)
    (0.9047380, 0.01388559)
    (0.8882062, 0.01593565)
    (0.8715962, 0.01795616)
    (0.8549446, 0.01994301)
    (0.8382694, 0.02189448)
    (0.8215796, 0.02380992)
    (0.8048802, 0.02568905)
    (0.7881741, 0.02753174)
    (0.7714632, 0.02933785)
    (0.7547492, 0.03110713)
    (0.7380334, 0.03283924)
    (0.7213171, 0.03453371)
    (0.7046016, 0.03618991)
    (0.6878882, 0.03780710)
    (0.6711780, 0.03938437)
    (0.6544725, 0.04092067)
    (0.6377730, 0.04241478)
    (0.6210807, 0.04386535)
    (0.6043972, 0.04527084)
    (0.5877237, 0.04662956)
    (0.5710618, 0.04793964)
    (0.5544131, 0.04919904)
    (0.5377791, 0.05040555)
    (0.5211615, 0.05155676)
    (0.5045622, 0.05265006)
    (0.4879829, 0.05368266)
    (0.4714257, 0.05465156)
    (0.4548929, 0.05555352)
    (0.4383865, 0.05638509)
    (0.4219094, 0.05714258)
    (0.4054640, 0.05782205)
    (0.3890536, 0.05841929)
    (0.3726814, 0.05892981)
    (0.3563512, 0.05934879)
    (0.3400671, 0.05967113)
    (0.3238340, 0.05989134)
    (0.3076573, 0.06000357)
    (0.2915433, 0.06000157)
    (0.2754992, 0.05987865)
    (0.2595337, 0.05962761)
    (0.2436573, 0.05924080)
    (0.2278824, 0.05871002)
    (0.2122249, 0.05802654)
    (0.1967044, 0.05718119)
    (0.1813469, 0.05616449)
    (0.1661868, 0.05496702)
    (0.1512710, 0.05358012)
    (0.1366645, 0.05199727)
    (0.1224568, 0.05021633)
    (0.1087684, 0.04824299)
    (0.09575211, 0.04609495)
    (0.08358167, 0.04380527)
    (0.07242320, 0.04142167)
    (0.06239496, 0.03899972)
    (0.05353650, 0.03659163)
    (0.04580557, 0.03423655)
    (0.03910103, 0.03195685)
    (0.03329449, 0.02976012)
    (0.02825571, 0.02764358)
    (0.02386683, 0.02559846)
    (0.02002796, 0.02361304)
    (0.01665780, 0.02167459)
    (0.01369190, 0.01977039)
    (0.01108039, 0.01788830)
    (0.008785779, 0.01601716)
    (0.006781116, 0.01414711)
    (0.005048412, 0.01227024)
    (0.003577171, 0.01038135)
    (0.002362721, 0.008479204)
    (0.001403676, 0.006567576)
    (0.0006990856, 0.004657032)
    (0.0002435545, 0.002761454)
    (0.00002599979, 0.0009056400)
    (0.00002599979, -0.0009056400)
    (0.0002435545, -0.002761454)
    (0.0006990856, -0.004657032)
    (0.001403676, -0.006567576)
    (0.002362721, -0.008479204)
    (0.003577171, -0.01038135)
    (0.005048412, -0.01227024)
    (0.006781116, -0.01414711)
    (0.008785779, -0.01601716)
    (0.01108039, -0.01788830)
    (0.01369190, -0.01977039)
    (0.01665780, -0.02167459)
    (0.02002796, -0.02361304)
    (0.02386683, -0.02559846)
    (0.02825571, -0.02764358)
    (0.03329449, -0.02976012)
    (0.03910103, -0.03195685)
    (0.04580557, -0.03423655)
    (0.05353650, -0.03659163)
    (0.06239496, -0.03899972)
    (0.07242320, -0.04142167)
    (0.08358167, -0.04380527)
    (0.09575211, -0.04609495)
    (0.1087684, -0.04824299)
    (0.1224568, -0.05021633)
    (0.1366645, -0.05199727)
    (0.1512710, -0.05358012)
    (0.1661868, -0.05496702)
    (0.1813469, -0.05616449)
    (0.1967044, -0.05718119)
    (0.2122249, -0.05802654)
    (0.2278824, -0.05871002)
    (0.2436573, -0.05924080)
    (0.2595337, -0.05962761)
    (0.2754992, -0.05987865)
    (0.2915433, -0.06000157)
    (0.3076573, -0.06000357)
    (0.3238340, -0.05989134)
    (0.3400671, -0.05967113)
    (0.3563512, -0.05934879)
    (0.3726814, -0.05892981)
    (0.3890536, -0.05841929)
    (0.4054640, -0.05782205)
    (0.4219094, -0.05714258)
    (0.4383865, -0.05638509)
    (0.4548929, -0.05555352)
    (0.4714257, -0.05465156)
    (0.4879829, -0.05368266)
    (0.5045622, -0.05265006)
    (0.5211615, -0.05155676)
    (0.5377791, -0.05040555)
    (0.5544131, -0.04919904)
    (0.5710618, -0.04793964)
    (0.5877237, -0.04662956)
    (0.6043972, -0.04527084)
    (0.6210807, -0.04386535)
    (0.6377730, -0.04241478)
    (0.6544725, -0.04092067)
    (0.6711780, -0.03938437)
    (0.6878882, -0.03780710)
    (0.7046016, -0.03618991)
    (0.7213171, -0.03453371)
    (0.7380334, -0.03283924)
    (0.7547492, -0.03110713)
    (0.7714632, -0.02933785)
    (0.7881741, -0.02753174)
    (0.8048802, -0.02568905)
    (0.8215796, -0.02380992)
    (0.8382694, -0.02189448)
    (0.8549446, -0.01994301)
    (0.8715962, -0.01795616)
    (0.8882062, -0.01593565)
    (0.9047380, -0.01388559)
    (0.9211172, -0.01181512)
    (0.9371981, -0.009743313)
    (0.9527169, -0.007706239)
    (0.9672663, -0.005761861)
    (0.9803692, -0.003981369)
    (0.9916796, -0.002421450)
    (1.000000, -0.001260000)
  };
\end{scope}
}

\newcommand\DrawCamberedAirfoil[1][]{
\begin{scope}[#1]
  % NACA 2412
  \draw plot [smooth] coordinates {
    (1.0, -0.0012599999999999777)
    (0.96549018468297, -0.003766550717226144)
    (0.930883045462546, -0.006219202442366447)
    (0.8961437971575432, -0.008631733865730334)
    (0.8613016381510661, -0.011011201185171479)
    (0.8263867428041894, -0.013362379670991803)
    (0.7914310164102236, -0.015687541221114883)
    (0.7564670891854666, -0.01798637927354734)
    (0.7215286223342008, -0.020255875905877893)
    (0.6866502816986508, -0.022490225407487845)
    (0.6518674215422855, -0.02468081055332231)
    (0.6172164265648663, -0.026816170723141183)
    (0.5827346534458342, -0.02888202916719134)
    (0.5484605052100948, -0.03086134225579657)
    (0.5144339818074576, -0.032734349703361115)
    (0.480696823454966, -0.03447867568513195)
    (0.4472934967551756, -0.03606940997738771)
    (0.41393567802086406, -0.037492644839771605)
    (0.3804149998506693, -0.03874755535699173)
    (0.3468272731008735, -0.039902691513794025)
    (0.3136190982592265, -0.04090782505712402)
    (0.2811814440525585, -0.04170125833576889)
    (0.2496338451339312, -0.04222939050878645)
    (0.21912534718162296, -0.04243729854617861)
    (0.1898439100344943, -0.042270895665989155)
    (0.16202614783530983, -0.0416809040510345)
    (0.13596390146008333, -0.04062966635981699)
    (0.11199933361948029, -0.039101582938232377)
    (0.09049621574701124, -0.037116349985295505)
    (0.07361924489626276, -0.0350073364191772)
    (0.05636259357037546, -0.0321514577872086)
    (0.04397649402900053, -0.02948274558741349)
    (0.031214918278790867, -0.025913353166771468)
    (0.025386870661493165, -0.02387293613730815)
    (0.01986852747692683, -0.021588341403196154)
    (0.015810475611440528, -0.019602135923702295)
    (0.012032895834446736, -0.01741586159249026)
    (0.008347783658642402, -0.014804213953756641)
    (0.006782776357419023, -0.013476228004680135)
    (0.005101987955964444, -0.01182609576308366)
    (0.004152048570182651, -0.010748129365261013)
    (0.0030603428943254468, -0.009316573595328595)
    (0.0023446574801605675, -0.008213642891790378)
    (0.001731083959066247, -0.007107510715816521)
    (0.0012037942711423102, -0.005969188316183293)
    (0.0006943488170150654, -0.00457210891942128)
    (0.0002949883265427625, -0.0030077870143021875)
    (0.00013268271827913483, -0.002028658161884189)
    (0.000047353525254730774, -0.0012175353029981746)
    (0.00002532476506966959, 0.0008970844059684918)
    (0.00018814793077252665, 0.002448072322554881)
    (0.00037457810981088356, 0.00345681778994653)
    (0.0006226843506845091, 0.004460305424217867)
    (0.0009189939842633151, 0.0054224385524775705)
    (0.0016420541089524005, 0.007257797625008717)
    (0.0028235547140623278, 0.00953206869809826)
    (0.0039902533393324745, 0.01134485186269217)
    (0.00508347301295101, 0.012816488263046943)
    (0.006234932100785785, 0.014205330332066607)
    (0.008164412430203494, 0.016273130533696094)
    (0.0109096354233833, 0.01883301025054221)
    (0.0149335517829051, 0.022059247763182688)
    (0.02090169719548933, 0.026119219613711435)
    (0.029090969152910873, 0.0308133496989451)
    (0.03936014651201021, 0.03579205910007571)
    (0.05135871826960699, 0.040760579490798525)
    (0.06527912321719298, 0.04572188569496259)
    (0.08128812301403121, 0.05064207626527274)
    (0.09937610494377608, 0.05542524513564439)
    (0.11929438296032827, 0.0599347534152128)
    (0.14075992658866143, 0.06406441454515942)
    (0.1635174601103429, 0.06774419165547137)
    (0.18735476998529443, 0.07093243648306191)
    (0.21210090281011207, 0.07360721154371101)
    (0.23761935376120738, 0.0757595916459023)
    (0.2638006186070965, 0.07738903218729433)
    (0.29055586446557025, 0.07850032882407872)
    (0.31781188174707375, 0.07910164537520417)
    (0.34550708677686826, 0.07920323908566278)
    (0.37365456652101114, 0.07881517686907744)
    (0.402631894340016, 0.07793063620654582)
    (0.43241230840392214, 0.07660997118115524)
    (0.46290547434324336, 0.07490494347854927)
    (0.4936649080396488, 0.07284825058170255)
    (0.5246548772457837, 0.0704561000377626)
    (0.5558421106966199, 0.06774368002361203)
    (0.5871951553016743, 0.06472507690681442)
    (0.6186840935139319, 0.061413195624036104)
    (0.650280234781535, 0.05781970701641605)
    (0.6819560363938716, 0.053954993285283706)
    (0.7136849190124785, 0.049828124742128076)
    (0.7454411587455343, 0.045446849321571406)
    (0.7771999395698983, 0.040817578741059804)
    (0.8089371985433962, 0.0359454208624503)
    (0.840629806069205, 0.030834179949050796)
    (0.8722552320885923, 0.025486452979646447)
    (0.903791999768359, 0.01990361393618675)
    (0.935219453987816, 0.014085919535834577)
    (0.9672766290279255, 0.007882534061003747)
    (1.0, 0.0012599999999999777)
  };
\end{scope}
}

\newcommand\DrawCenterOfMass[1][]{
  \begin{scope}[#1]
    \fill (0, 0) -- (1, 0) arc [radius=1, start angle=0, end angle=90];
    \fill (0, 0) -- (0, -1) arc [radius=1, start angle=270, end angle=180];
    \draw (0, 0) circle [radius=1];
  \end{scope}
}

\newcommand\DrawWingFrontWithCoordinateSystem[1][]{
  \begin{scope}[#1]
    \DrawCenterOfMass[scale=0.3]
    \draw[dashed, ->] (0, 0) -- (-2, 0) node[above right] {$y_b$};
    \draw[dashed, ->] (0, 0) -- (0, -2) node[above left] {$z_b$};
  \end{scope}
  \DrawWingFront[#1]
}

\newcommand\DrawWingFront[1][]{
  \begin{scope}[#1]
    % Main wing.
    \draw plot [smooth] coordinates {
      (-12.65, 1.45)
      (-12.43, 0.40)
      (-9.5, 0.23)
      (-6.43, 0.07)
      (-3.36, 0.07)
      (-0.3, 0.07)
      (0.3, 0.07)
      (3.36, 0.07)
      (6.43, 0.07)
      (9.5, 0.23)
      (12.43, 0.40)
      (12.65, 1.45)
      (12.85, 1.25)
      (12.43, 0.20)
      (9.5, 0.03)
      (6.43, -0.13)
      (3.36, -0.13)
      (0.3, -0.13)
      (-0.3, -0.13)
      (-3.36, -0.13)
      (-6.43, -0.13)
      (-9.5, 0.03)
      (-12.43, 0.20)
      (-12.85, 1.25)
      (-12.65, 1.45)
    };

    % Horizontal tail.
    \draw (-2.44, 3.55) rectangle (2.47, 3.65);

    % Vertical tail.
    \draw (-0.05, 0.0) rectangle (0.05, 3.6);

    % Pylons.
    \draw (3.55, -1.60) rectangle (3.75, 1.22);
    \draw (1.15, -1.60) rectangle (1.3, 1.22);
    \draw (-1.15, -1.60) rectangle (-1.3, 1.22);
    \draw (-3.55, -1.60) rectangle (-3.75, 1.22);

    % Rotors.
    \draw (-3.64, -1.60) circle [radius=1.1];
    \draw (-1.21, -1.60) circle [radius=1.1];
    \draw (1.21, -1.60) circle [radius=1.1];
    \draw (3.64, -1.60) circle [radius=1.1];
    \draw (-3.64, 1.22) circle [radius=1.03];
    \draw (-1.21, 1.22) circle [radius=1.03];
    \draw (1.21, 1.22) circle [radius=1.03];
    \draw (3.64, 1.22) circle [radius=1.03];
  \end{scope}
}

\newcommand\DrawWingSideWithCoordinateSystem[1][]{
  \begin{scope}[#1]
    \DrawCenterOfMass[scale=0.3]
    \draw[dashed, ->] (0, 0) -- (0, 2) node[above right] {$x_b$};
    \draw[dashed, ->] (0, 0) -- (-2, 0) node[above left] {$z_b$};
  \end{scope}
  \DrawWingSide[#1]
}

\newcommand\DrawWingSide[1][]{
  \begin{scope}[#1]
    % Main wing.
    \DrawCamberedAirfoil[shift={(0, 0.36)}, rotate=-90, scale=1.46]

    % Pylons.
    \draw plot [smooth] coordinates {
      (1.4510, 1.9207)
      (1.1927, 2.3578)
      (0.9825, 1.8961)
      (0.8782, 1.4807)
      (0.0133, 1.3385)
      (-0.0485, 1.3280)
      (-0.4537, 1.2653)
      (-1.2008, 1.1512)
      (-1.3639, 1.5729)
      (-1.6213, 2.0100)
      (-1.8324, 1.5484)
      (-1.8324, 0.1574)
      (-1.6213, 0.1567)
      (-1.3639, 0.1029)
      (-1.2008, 0.0600)
      (-0.4537, -0.4500)
      (-0.0485, -0.4500)
      (0.0133, 0.5645)
      (0.8782, 0.5163)
      (0.9825, 0.5044)
      (1.1927, 0.5054)
      (1.4510, 0.5156)
      (1.4510, 1.9207)
    };

    % Propellers.
    \draw (-1.597 - 1, 1.613) -- (-1.597 + 1, 1.613);
    \draw (1.216 - 1, 1.960) -- (1.216 + 1, 1.960);

    % Fuselage.
    \draw plot [smooth] coordinates {(0, 0) (0, -3) (1.42, -6.7)};

    % H tail.
    \DrawAirfoil[shift={(3.6, -6.25)}, rotate=-90, scale=0.99]

    % V tail.
    \draw (3.45, -6.86) -- (3.45, -7.62) -- (1.42, -7.72) -- (0.0, -7.62) --
          (0.0, -6.89) -- (1.42, -6.70) -- (3.45, -6.86);
  \end{scope}
}

\newcommand\DrawWingBottomWithCoordinateSystem[1][]{
  \begin{scope}[#1]
    \DrawCenterOfMass[scale=0.3]
    \draw[dashed, ->] (0, 0) -- (0, 2) node[above right] {$x_b$};
    \draw[dashed, ->] (0, 0) -- (-2, 0) node[above left] {$y_b$};
  \end{scope}
  \DrawWingBottom[#1]
}

\newcommand\DrawWingBottom[1][]{
  \begin{scope}[#1]
    % Pylons.
    \DrawCamberedAirfoil[shift={(-3.64, 1.32)}, rotate=-90, scale=1.78]
    \DrawCamberedAirfoil[shift={(-1.21, 1.32)}, rotate=-90, scale=1.78]
    \DrawCamberedAirfoil[shift={(1.21, 1.32)}, rotate=-90, scale=1.78]
    \DrawCamberedAirfoil[shift={(3.64, 1.32)}, rotate=-90, scale=1.78]

    % Fuselage.
    \draw (0, 0) -- (0, -6.25);

    % Main wing.
    \draw (-12.85, 0.09) -- (-12.85, -0.58) -- (-9.50, -0.85) --
          (-6.43, -1.1) -- (-0.3, -1.1) -- (0.3, -1.1) -- (6.43, -1.1) --
          (9.50, -0.85) -- (12.85, -0.58) -- (12.85, 0.09) -- (9.50, 0.28) --
          (6.43, 0.36) -- (0.3, 0.36) -- (-0.3, 0.36) -- (-6.43, 0.36) --
          (-9.50, 0.28) -- (-12.85, 0.09);

    % H-Tail.
    \draw (-2.43, -6.68) -- (-2.43, -7.11) -- (2.43, -7.11) --
          (2.43, -6.68) -- (0, -6.25) -- (-2.43, -6.68);

    % V-tail.
    \DrawAirfoil[shift={(0, -6.9)}, rotate=-90, scale=1.03]
  \end{scope}
}

\newcommand{\DrawCoordinateSystem}[4] {
  \begin{scope}[#1]
    \draw[dashed, ->] (0, 0) -- (-1, 0) node[below] {#2};
    \draw[dashed, ->] (0, 0) -- (0, -1) node[right] {#4};
    \draw (0, 0) node[below right] {#3};
    \begin{scope}[rotate=45, scale=0.1]
      \draw (0, 0) circle (1);
      \draw (-1, 0) -- (1, 0);
      \draw (0, -1) -- (0, 1);
    \end{scope}
  \end{scope}
}

\newcommand{\DrawCoordinateSystemDot}[4] {
  \begin{scope}[#1]
    \draw[dashed, ->] (0, 0) -- (-1, 0) node[below] {#2};
    \draw[dashed, ->] (0, 0) -- (0, -1) node[right] {#4};
    \draw (0, 0) node[below right] {#3};
    \begin{scope}[rotate=45, scale=0.1]
      \draw (0, 0) circle (1);
      \draw[black, fill=black] (0, 0) circle (0.33);
    \end{scope}
  \end{scope}
}

\makeatletter
\dspdeclareoperator{dspvoidshapeadder}{
  % Coordinate offset for the plus
  \pgfutil@tempdima=\radius
  \pgfutil@tempdima=0.55\pgfutil@tempdima
  \pgfusepathqstroke
}
\tikzset{
  vdspadder/.style={
    shape=dspvoidshapeadder,
    line cap=rect,
    line join=rect,
    line width=\dspblocklinewidth,
    minimum size=\dspoperatordiameter,
    label={185:$+$},
    label={265:$-$}
  },
  vadspadder/.style={
    shape=dspvoidshapeadder,
    line cap=rect,
    line join=rect,
    line width=\dspblocklinewidth,
    minimum size=\dspoperatordiameter,
    label=below right:$-$,
    label=above right:$+$
  },
  vkdspadder/.style={
    shape=dspvoidshapeadder,
    line cap=rect,
    line join=rect,
    line width=\dspblocklinewidth,
    minimum size=\dspoperatordiameter,
    label=below right:$-$,
    label=below left:$+$
  },
  vjdspadder/.style={
    shape=dspvoidshapeadder,
    line cap=rect,
    line join=rect,
    line width=\dspblocklinewidth,
    minimum size=\dspoperatordiameter,
    label=above right:$+$,
    label=below right:$-$
  },
  edspadder/.style={
    shape=dspvoidshapeadder,
    line cap=rect,
    line join=rect,
    line width=\dspblocklinewidth,
    minimum size=\dspoperatordiameter
  }
}
\makeatother


\tikzset{dspblock/.style={inner xsep = 5pt, inner ysep = 5pt}}
\tikzset{dspblock2/.style={inner xsep = 5pt, inner ysep = 10pt}}
\tikzset{dspblock3/.style={inner xsep = 5pt, inner ysep = 20pt}}


\newcommand{\aero}{\mathrm{aero}}
\newcommand{\airgrid}{\mathrm{air-grid}}
\newcommand{\app}{a}
\newcommand{\climb}{\mathrm{climb}}
\newcommand{\cmd}{\mathrm{cmd}}
\newcommand{\cutin}{\mathrm{cut-in}}
\newcommand{\cutout}{\mathrm{cut-out}}
\newcommand{\cw}{\mathrm{cw}}
\newcommand{\eff}{\mathrm{eff}}
\newcommand{\extra}{\mathrm{extra}}
\newcommand{\figureeight}{\mathrm{figure-eight}}
\newcommand{\grav}{\mathrm{grav}}
\newcommand{\groundinverter}{\mathrm{ground\;inv.}}
\newcommand{\hover}{\mathrm{hover}}
\newcommand{\kiteinverter}{\mathrm{kite\;inv.}}
\newcommand{\kite}{\mathrm{kite}}
\newcommand{\lp}{\mathrm{loop}}
\newcommand{\mx}{\mathrm{max}}
\newcommand{\mn}{\mathrm{min}}
\newcommand{\motor}{\mathrm{motor}}
\newcommand{\nominal}{\mathrm{nom}}
\newcommand{\path}{\mathrm{path}}
\newcommand{\prop}{\mathrm{prop}}
\newcommand{\shaftgrid}{\mathrm{shaft-grid}}
\newcommand{\swept}{\mathrm{swept}}
\newcommand{\tether}{\mathrm{tether}}
\newcommand{\total}{\mathrm{tot}}
\newcommand{\wind}{\mathrm{wind}}
\newcommand{\mass}{\mathrm{mass}}
\newcommand{\power}{\mathrm{power}}
\newcommand{\powertrain}{\mathrm{powertrain}}
\newcommand{\motorout}{\mathrm{motor-out}}

\definecolor{matlab1}{rgb}{0, 0.4470, 0.7410}
\definecolor{matlab2}{rgb}{0.8500, 0.3250, 0.0980}
\definecolor{matlab3}{rgb}{0.9290, 0.6940, 0.1250}
\definecolor{matlab4}{rgb}{0.4940, 0.1840, 0.5560}
\definecolor{matlab5}{rgb}{0.4660, 0.6740, 0.1880}
\definecolor{matlab6}{rgb}{0.3010, 0.7450, 0.9330}
\definecolor{matlab7}{rgb}{0.6350, 0.0780, 0.1840}

\title{Limits on kite mass}
\author{Makani Technologies LLC}
\date{October 6, 2016}

\begin{document}
\maketitle

% \section*{Nomenclature}

% \begin{tabular}{ll}
%   $m_{\kite}$       & Kite mass \\
%   $m_{\total}$      & Combined kite and tether mass \\
%   $m_{\eff}$        & Effective mass ($m_{\eff} = m_{\kite} + m_{\tether}/3)$ for modeling dynamics \\
%   $\mathit{TWR}$    & Thrust-to-weight ratio \\
%   $\eta_{\airgrid}$ & Efficiency of power conversion from the grid to aerodynamic power \\

% \end{tabular}

\section{Vertical take-off and landing}

The relationship between the required grid power for hover and mass
is:
%
\begin{equation}
  \label{eqn:hover_power}
  P_{\hover} = \frac{1}{\eta_{\shaftgrid} \cdot \mathit{FM}}
               \sqrt{\frac{(\mathit{TWR} \cdot m_{\total} g)^3}
                          {2 \rho_{\mn} f N_{\prop} A_{\prop}}}
\end{equation}
%
where $\mathit{FM}$ is the figure-of-merit of the propellers,
$\mathit{TWR}$ is the thrust-to-weight ratio, $f$ is the fraction of
motors available in a motor-out situation, and $m_{\total}$ is the
combined kite and tether mass.\footnote{The strict limits that
  vertical take-off and landing places on mass and power have
  been covered extensively in other documents
  \cite{tvanalsenoy_power_train}}

Because of the significant cost of the ground-side inverter and other
power components, which scales approximately linearly with power,
vertical take-off and landing (VTOL) systems likely do not make
economic sense unless $P_{\hover} \le P_{\mx}$.  This constraint may
be relaxed somewhat for multi-kite systems, if it is only necessary to
hover a single kite at a time.  In this case, the kite is likely
limited by the available shaft power from the motors.  In a
well-designed system, the available shaft power from the motors should
be approximately, $P_{\mx} / \eta_{\shaftgrid}$, where $P_{\mx}$ is
the per-kite rated power, so that the power system is fully utilized
during generation.  In this case, the constraint imposed by VTOL is:
$\eta_{\shaftgrid}^2 P_{\hover} / f \le P_{\mx}$.  Note that the extra
factor of $\eta_{\shaftgrid} / f$ accounts for the fact that
Eq. \ref{eqn:hover_power} is written in terms of ground power, and
also that it is necessary to acccount for the motor-out fraction again
in a shaft power limited system.

For a single kite M600-like system, assuming $P_{\mx} = 600$ kW,
$\eta_{\shaftgrid} = 0.83$, $\mathit{FM} = 0.68$, $\rho_{\mn} = 0.95$
kg/m$^3$, $A_{\prop} = 3.8$ m$^2$, and $\mathit{TWR} = 1.3$,
$f = 3/4$, this sets a reasonable mass limit of $m_{\total} < 1338$
kg.  For a multi-kite M600-like system, with equivalent parameters,
the mass limit is $m_{\total} < 1416$ kg.  If the motor-out fraction
is increased to $f = 7/8$, then the single- and multi-kite limits are
1409 kg and 1652 kg, respectively.  Above these mass limits, VTOL
becomes expensive quickly, scaling as
$P_{\hover} \propto (m/m_{\nominal})^{3/2}$.

It is of course possible to increase the VTOL lifting ability through
significant design changes to the M600.  The most significant
parameter to change is rotor disk area.  In particular,
Eq. \ref{eqn:hover_power} can be used to solve for the rotor disk area
that meets a given power requirement.  Later in this paper, limits on
the power density of a kite system will be derived, so it is actually
more useful to reexpress Eq. \ref{eqn:hover_power} as a relationship
between maximum power density and disk mass loading:
%
\begin{equation}
  \label{eqn:hover_power2}
  \left(\frac{P_{\mx}}{m_{\total}}\right)^2 >
  k \left(\frac{m_{\total}}{N_{\prop} A_{\prop}}\right)
\end{equation}
%
Assuming the same values as above, the proportionality constant here
is $k \approx 6400$ W$^2$-m$^2$/kg$^3$.  To get a feel for the range
of values this constant could take across kite systems, the value
without motor-out capability is $k \approx 2700$ W$^2$-m$^2$/kg$^3$.

\section{Crosswind flight}

Crosswind flight does not place as strict of a limit on mass as VTOL;
however it still imposes mass limitations stricter than what would be
required for traditional flight vehicles.  The main consequences of
increased mass in crosswind flight are 1) an increase in power
required to maintain a minimum airspeed during climb, 2) an increase
in minimum turning radius, and 3) an increase in power and tension
variations around a loop.  Each of these consequences are analyzed in
detail below.

\subsection{Minimum airspeed climb}

A reasonable requirement for a kite system is to be able to maintain
the minimum allowed airspeed, $v_{\app,\mn}$, while climbing up the loop
with low or zero wind.  Assuming a circular flight path, this results
in the power requirement for climbing of
%
\begin{equation}
  P_{\climb} = \frac{m_{\eff} g_{\parallel} v_{\app,\mn}}{\eta_{\airgrid}}
\end{equation}
%
which may be converted to a limitation on the maximum reasonable
effective mass:
%
\begin{equation}
  \label{eqn:mass_minimum_airspeed_climb}
  m_{\eff} < \frac{\eta_{\airgrid} P_{\mx}}{g_{\parallel} v_{\app,\mn}}
\end{equation}
%
For an M600-like system, assuming $P_{\mx} = 600$ kW,
$\eta_{\airgrid} = 0.67$, a flight path elevation angle of
25$^{\circ}$, and $v_{\app,\mn} = 30$ m/s, the maximum reasonable
effective mass is $m_{\eff} = 1507$ kg.\footnote{TODO: Add drag to
  calculation.}

\begin{figure}[h]
\begin{center}
  \begin{tikzpicture}
    \begin{scope}[scale=2]]
      % Downstrokes.
      \draw[arrows={latex-}] ({sqrt(2) + 1}, 0) arc (0:135:1);
      \draw                  ({sqrt(2) + 1}, 0) arc (0:-135:1);

      \draw[arrows={latex-}] ({-(sqrt(2) + 1)}, 0) arc (180:45:1);
      \draw                  ({-(sqrt(2) + 1)}, 0) arc (180:315:1);

      % Upstrokes.
      \draw[arrows={-latex}] ({-sqrt(2)/2}, {-sqrt(2)/2}) --
      ({ sqrt(2)/2}, { sqrt(2)/2});
      \draw[arrows={-latex}] ({ sqrt(2)/2}, {-sqrt(2)/2}) --
      ({-sqrt(2)/2}, { sqrt(2)/2});

      \draw ({-sqrt(2) - 0.1}, 0) -- ({-sqrt(2) + 0.1}, 0);
      \draw ({-sqrt(2)}, -0.1) -- ({-sqrt(2)}, 0.1);

      \draw ({sqrt(2) - 0.1}, 0) -- ({sqrt(2) + 0.1}, 0);
      \draw ({sqrt(2)}, -0.1) -- ({sqrt(2)}, 0.1);

      \draw[dashed] (0, 0) -- ({sqrt(2)}, 0);
    \end{scope}

    \draw[arrows={latex-latex}] (1, 0) arc (0:45:1);
    \path (1, 0) arc (0:22.5:1) node[above right] {$\psi_{\climb}$};

    \draw [decorate, decoration={brace, amplitude=5pt}]
      ({-sqrt(2)*2}, 0) -- ({-sqrt(2)*2 - 1}, {sqrt(3)})
      node[midway, below left] {$R_{\path}$};

    \draw [decorate, decoration={brace, amplitude=5pt}]
      (0, 0) -- ({-(sqrt(2) * 2)}, 0)
      node[midway, below=0.2] {$\delta$};

  \end{tikzpicture}
  \caption{Figure-eight flight path.}
\end{center}
\end{figure}

One concept that attempts to relax the constraints imposed by a
minimum airspeed climb is to fly figure-eights where the upstrokes are
the long paths between downstrokes.  Now the power to climb is reduced
by the sine of the climb angle:
%
\begin{equation}
  \label{eqn:mass_minimum_airspeed_climb2}
  P_{\climb} = \frac{m_{\eff} g_{\parallel} v_{\app,\mn} \sin \psi_{\climb}}
                    {\eta_{\airgrid}}
\end{equation}
%
Unfortunately, the figure-eight pattern has significant negative
effects on power generation.  One effect is that it pushes part of the
path off-downwind.  Specifically, the center of each of the loops of
the eight will be approximately
$\delta = R_{\path} / \sin \psi_{\climb}$ off-downwind.  Thus, there
will be a power loss roughly proportional to
%
\begin{equation}
  \eta_{\figureeight} \sim \cos^3 (\sin^{-1}(\sin \gamma / \sin \psi_{\climb}))
\end{equation}
%
where $\gamma$ is the half-cone angle of the flight path.  Even for
moderate climb angles, say $\psi_{\climb} = 45^{\circ}$, these losses
are substantial, $\eta_{\figureeight} \sim 0.67$ assuming
$\gamma = 20^{\circ}$ \footnote{For single kite systems, the required
  low elevation angle of the flight path puts a strict limit on the
  maximum half-cone angle.  This can be relaxed somewhat for
  multi-kite systems.}.  And the estimated efficiency drops off
quickly with shallower climb angles, $\eta_{\figureeight} = 0.38$ at
$\psi_{\climb} = 30^{\circ}$.  A second slightly harder to quantify
effect is that the figure-eight will take longer to fly than the
equivalent set of two independent circles, which also reduces power
efficiency.

Nonetheless, flying figure-eights does offer an approach to increasing
the maximum mass calculated with
Eq. \ref{eqn:mass_minimum_airspeed_climb} by a factor of, very
roughly, 1.5, while sacrificing some power efficiency.  For an
M600-like system flying figure-eights, the maximum reasonable
effective mass is, again very roughly, 2300 kg.

Equations \ref{eqn:mass_minimum_airspeed_climb} and
\ref{eqn:mass_minimum_airspeed_climb2} set a fairly strict limit on
the power density of kite systems, which is independent of system
scale.  Using the same parameters used above for the M600 system,
which should apply equally well to a generic kite system, the power
density for a kite system flying a circular path must be greater than
400 W/kg, and the power density for a kite system flying a
figure-eight must be greater than roughly 265 W/kg.

\subsection{Turning radius}

The ability of a kite to turn around a tight circular path is
inversely proportional to its mass.  Higher masses will eventually
force either a larger path radius or an excessively large roll angle.
A larger path radius leads to a longer, draggier, and heavier tether
as well as larger swings in tension and power.  Similarly, an
excessively large roll angle has negative consequences due to the
reduced projected wing area normal to the wind.

To quantify these effects, it is useful to derive the exact
relationship between wing mass and turning radius.  This relationship
can be found by balancing forces in the stability axes' y-z plane.  To
simplify the calculation, the tether is appoximated as a straight line
from the ground-station to the bridle point.  Other effects on the
tether such as the catenary from gravity are ignored.

\begin{figure}[h]
\begin{center}
  \begin{tikzpicture}[scale=0.5]
    \begin{scope}[shift={(-5, 0)}, scale=0.3, rotate=-10]
      \DrawWingFront[]
      \draw[line width=1.5pt, -latex] (4, 1) -- (9, 1) node[midway, above] {$-Y$};
      \draw[line width=1.5pt, -latex] (0, 5) -- (0, 12) node[midway, right] {$L$};
      \draw (-5.86, -0.13) -- (0, -4.92);
      \draw (5.86, -0.13) -- (0, -4.92);
      \draw[dashed] (0, -4.92) -- (0, -15);
    \end{scope}
    \node at (-5, -3) {$\phi_t$};
    \draw[line width=1.5pt, -latex] (-5.25, -1.45) --
    (-3.5, -4.3) node[midway, above right] {$t$};
    \draw[line width=1.5pt, -latex] (-3.5, -1) -- (-2, -2)
    node[below right] {$W$};
    \draw (-5.25, -1.45) -- (0, -10) node[midway, above right] {$l_t$};
    %\draw[dashed] (-5.25, -1.45) -- (-0.975, 1.175);
    \node at (-0.5, -8.5) {$\gamma$};
    \draw[dashed] (0, -10) -- (0, -5);
    %\draw[dashed] (-5, 0) -- (-10, 0) node[near end, above] {$-\phi$};
    %\draw[line width=1.5pt, -latex] (-7, 0) -- (-9, 0) node[midway, below]
    %{$mv^2/R$};
    \DrawCoordinateSystem{shift={(0, -10)}, scale=1}
                         {$x_{\cw}$}{$y_{\cw}$}{$z_{\cw}$}
  \end{tikzpicture}
  \caption{Diagram of force balance in the stability axes' y-z plane.}
\end{center}
\end{figure}

It is easiest to conduct the force balance along axes that are
parallel and perpendicular to the tether.  Because the wing is forced
to fly on the surface of a sphere with radius equal to the tether
length, $l_t$, the acceleration parallel to the tether is
$a_{\parallel} = v_i^2/l_t$.  The lift, pylon side force, and weight
each have components along the direction perpendicular to the tether.
Together, these create an acceleration perpendicular to the tether
given by
%
\begin{equation}
m_{\eff} a_{\bot} = L \sin \phi_t - Y \cos \phi_t -
W_{\cw, x} \cos \gamma - W_{\cw, z} \sin \gamma
\end{equation}
%
Here $m_{\eff}$ is the effective mass of the kite, which is given by
$m_{\eff} = m_{\kite} + m_{\mathrm{tether}} / 3$.  The extra term from
the tether mass comes from calculating the acceleration of a point
mass at the end of a rigid rod due to a force applied at the point
mass.  Also, a few small approximations were made such as ignoring
forces from motor thrust, ignoring the component of drag along the
stability z-axis, and ignoring the cosine term from projecting the
lift vector onto the stability z-axis.

The acceleration projected onto the crosswind flight plane is the
acceleration relevant for following a circular path.  The parallel and
perpendicular accelerations can be combined, projected onto the
crosswind $x$-axis as follows:
%
\begin{align}
  \label{eqn:full_curvature}
  \frac{m_{\eff} v_i^2}{r} =&\;
    \frac{m_{\eff} v_i^2}{l_t} \sin \gamma +
    \frac{1}{2} \rho A v_{\app}^2 \left(C_L \sin \phi_t - C_Y \cos \phi_t \right) \cos \gamma \\ \notag
  & - m (g_{\cw,x} \cos \gamma +
  g_{\cw,z} \sin \gamma) \cos \gamma
\end{align}
%
For the purposes of this discussion, Eq. \ref{eqn:full_curvature} may
be simplified for the turn about the bottom of the loop, directly
downwind, while ignoring some of the smaller terms:
%
\begin{equation}
  r \approx \frac{m_{\eff} v_i^2}
                 {\frac{1}{2} \rho A v_{\app}^2 C_L \sin \phi_t \cos \gamma -
                 m g_{\parallel} \cos^2 \gamma}
\end{equation}


The tether roll angle, $\phi_t$, is the primary control variable for
turning the kite around the path.  It is also possible to use $C_L$ or
$C_Y$ to turn, but these variables significantly affect power
production, so they are likely set by other considerations.  At the
absolute minimum, it is necessary to be able to set $\phi_t$ such that
$r = R_{\path}$ to fly a circle.  However, to be able to {\it control}
the kite it is necessary to have an available-to-desired centripetal
force ratio, $\mathit{CFR}$, greater than one, a similar concept to a
thrust-to-weight ratio.  Thus, the constraint is that there is some
$\phi_t$ such that $1/r > \mathit{CFR} / R_{\path}$.  A realistic
centripetal force ratio may be $\mathit{CFR} \sim 1.5$.  The above
considerations result in the following constraint on path radius:
%
\begin{equation}
  R_{\path} > \mathit{CFR} \cdot
              \frac{m v_{i,\mn}^2}
                   {\frac{1}{2} \rho v_{i,\mn}^2 C_L A \sin \phi_{t,\mx} \cos \gamma_{\mx} -
                    m g_{\parallel} \cos^2 \gamma_{\mx}}
\end{equation}
%
Note that the apparent wind speed was replaced by an inertial speed
because it should still be possible to turn without wind!

The constraint on path radius can be converted to a constraint on
minimum tether length for a given kite mass:
%
\begin{equation}
  l_{t,\mn} > l_{t_0} \frac{m}{m_0 - m}
\end{equation}
%
where the characteristic tether length $l_{t_0}$ is:
%
\begin{equation}
  l_{t_0} = \mathit{CFR} \cdot
  \frac{m_{\eff} v_{i,\mn}^2}{m g_{\parallel} \sin(\gamma_{\mx}) \cos^2(\gamma_{\mx})}
\end{equation}
%
and the characteristic mass $m_0$ is:
%
\begin{equation}
  \label{eqn:characteristic_mass}
  m_0 = \frac{\frac{1}{2} \rho v_{i,\mn}^2 A C_L \sin(\gamma_{\mx} + \phi_{\mx})}
             {g_{\parallel} \cos(\gamma_{\mx})}
\end{equation}

Clearly, this gives an absolute hard limit on mass of $m < m_0$.  This
limit can be tightened somewhat by additionally specifying that the
derivative of the {\it minimum} tether length with respect to
effective mass should not grow faster than the derivative of the
tether length with respect to effective mass:
%
\begin{equation}
  m < m_0 - \sqrt{\frac{l_{t_0} m_0 \lambda_t}{3}}
\end{equation}
%
where $\lambda_t$ is the linear mass density of the tether.

For an M600-like system, assuming $\mathit{CFR} = 1.5$, $\rho = 0.95$
kg/m$^3$, $v_{i,\mn} = 30$ m/s, $\gamma_{\mx} = 20^{\circ}$,
$\phi_{\mx} = 30^{\circ}$, $\lambda_t = 1$ kg/m, and a flight path
elevation angle of $25^{\circ}$, $m_0 \approx 2700$ kg and
$l_{t_0} \approx 540$ m. This implies that $m_{\kite} < 2000$ kg.
And, ideally $m_{\kite}$ should be significantly less to avoid a
runaway in mass.

Equation \ref{eqn:characteristic_mass} sets a fairly strict limit on
wing mass loading, which is independent of system scale.
Specifically, to be able to fly in a circle on a reasonable length
tether, a kite system must have a wing mass loading of less than 60
kg/m$^2$.

\subsection{Tension and power variation}
This section is not finished yet.

% \section{Designing a kite}


% \begin{table}[h]
%   \begin{tabular}{cccc}
%     \hline
%     \hline
%     Parameter               & Symbol               & Value              & Units \\
%     \hline
%     Min. power density      & $\rho_{\power}$      & 400                & W/kg \\
%     Max. wing mass loading  & $\sigma_{\mass}$     & 60                 & kg/m$^2$ \\
%     M600 powertrain density & $\rho_{\powertrain}$ & $1.25 \times 10^3$ & W/kg \\
%     \hline
%     \hline
%   \end{tabular}
% \end{table}

% Choose a rated power: $P_{\mx} = 600$ kW.

% Use the minimum power density to find the maximum effective mass: $P_{\mx} / \rho_{\power} = 1500$ kg.

% Select a lower mass for margin: $m_{\eff, \mx} = 1200$ kg.

% Estimate the amount of kite mass: $m_{\kite, \mx} = 1100$ kg.

% Estimate the total mass: $m_{\total, \mx} = 1400$ kg.

% Use Eq. \ref{} to estimate total rotor area: $N_{\prop} A_{\prop} > 48.8$ m$^2$.

% For current motors, $P_{\mx} = 600$ kW implies $N_{\prop} = 8$.

% Thus, $A_{\prop} = 6.1$ m$^2$ and $R_{\prop} = 1.4$ m.

% Use maximum wing mass loading to determine a minimum wing area:
% $m_{\kite, \mx} / \sigma_{\mass} = 18$ m$^2$.

% Select a higher area for margin: $A_{\kite} = 25$ m$^2$.


\section{Conclusions}

\begin{itemize}

\item For a system with the same aerodynamic and geometric properties
  as the M600, VTOL imposes a {\it total} mass limit of 1200 kg; the
  minimum airspeed climb imposes an {\it effective} mass limit of 1500
  kg (this increases to 2300 kg for a figure-eight flight path); and
  the turning radius imposes a {\it kite} mass limit of 2000 kg.

\item According to the minimum airspeed climb criterion, a kite that
  flies circles must have a power density, referenced to {\it
    effective} mass, of greater than 400 W/kg.  A kite that flies
  figure-eights may relax this to roughly 265 W/kg.  A kite with a
  ground-based power system (i.e. pumping kite) can relax these limits
  even further, potentially to 200 W/kg for figure-eights, due to a
  more efficient power conversion process. These limits are
  essentially independent of scale.

\item Given that the specific power of our powertrains is ~1.25 kW/kg,
  approximately 1/3 to 1/2 of the system weight will be the
  powertrain, independent of scale.  For an M600-like VTOL system,
  this leaves $\sim 500$ kg for structure, servos, etc.

\item After a power density is selected, which meets the limit above,
  Eq. \ref{eqn:hover_power2} may be used to size the total rotor area.

\item According to the turning radius criterion, a kite must have a
  wing mass loading, referenced to {\it kite} mass, of less than 60
  kg/m$^2$.  Again, this limit is independent of scale.

\item At the 600 kW scale, designing a system that supports VTOL will
  require substantially reducing the maximum tension.  Because of the
  square-cube law, larger scales will likely require significant
  topology changes (e.g. more bridling, tailless variants, or
  multi-kites) coupled with a reduction in maximum tension.
\end{itemize}


\begin{thebibliography}{1}
\bibitem{tvanalsenoy_power_train} Alsenoy, Thomas.
  ``Airborne wind powertrain trade-off study.''
  \texttt{Drive > Makani Teams > Engineering > 01\_System\_Engineering > Papers (internal)}. 2016.
\end{thebibliography}

\end{document}
